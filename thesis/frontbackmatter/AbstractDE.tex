%*******************************************************
% Abstract in German
%*******************************************************
\begin{otherlanguage}{ngerman}
	\pdfbookmark[1]{Zusammenfassung}{Zusammenfassung}
	\chapter*{Zusammenfassung}
	Das Internet der Dinge (engl. IOT; Internet of Things) erhält immer mehr Einzug in das tägliche Leben. Smart-Home Lösungen, vernetzte und latenzempfindliche Connected-Cars oder die Vision einer Smart-City prägen die Forschungsarbeiten in den jeweiligen Bereichen. Ziel ist eine vollautomatische Machine-to-Machine (M2M) Abwicklung von Prozessen, um unseren Alltag zu automatisieren und zu vereinfachen. Dabei fallen eine Menge Daten an, die verarbeitet, übertragen und gespeichert werden müssen. Mit der Einführung des neuen Mobilfunk-Standards 5G und immer leistungsfähigeren Endgeräten sind Übertragung und Verarbeitung der Daten weitestgehend gesichert; bleibt die Frage offen, wo diese großen Datenmengen gespeichert und weiterprozessiert werden.
	Daneben erfreut sich das Thema Distributed-Ledger-Technology (DLT) immer größerer Beliebtheit: Es werden täglich neue Anwendungsfälle gefunden, die durch die verteilte Infrastruktur, der Trustless-Eigenschaft und der Dezentralität profitieren. Es bietet sich eine Untersuchung an, um zu überprüfen, inwieweit diese beiden noch recht jungen Technologien IOT und DLT Synergien besitzen und sich gegebenenfalls gegenseitig ergänzen können.
	Die vorliegende Masterarbeit evaluiert eine Auswahl etablierter DLTs anhand ihrer Tauglichkeit für den Einsatz im IOT-Umfeld mit Fokus auf den M2M-Bereich. Dazu wird zunächst ein IOT-Anwendungsfall erstellt, der stellvertretend für den M2M-Bereich für die weiteren Analysen verwendet wird. Anschließend werden konkrete Anforderungen aus verschiedenen Bereichen Infrastruktur, IT-Security, Performance und weiteren aufgestellt, die eine DLT erfüllen muss, um den Anforderungen des beispielhaften Anwendungsfalls gerecht zu werden. Die erstellten Kriterien werden auf eine Auswahl von DLT-Implementierungen angewandt, evaluiert und bewertet. Mit der am besten geeigneten DLT wird eine prototypische Implementierung des Anwendungsfalls vorgenommen, um die Ergebnisse aus der Anforderungsevaluierung zu überprüfen. Um den Use-Case möglichst realistisch zu simulieren werden Daten aus verschiedenen IOT-Sensoren an die DLT übermittelt und eine M2M-Komminkation zwischen IOT-Devices via DLT erstellt. Anschließende Load-Tests geben detaillierte Informationen über die Performance.
	Das Ergebnis ist eine strukturierte und nachvollziehbare Bewertung mehrerer, am Markt etablierter DLTs, inwieweit diese für DLT-sinnvolle IOT-Anwendungsfälle im M2M-Umfeld geeignet sind, sowie ein DLT-basierter Prototyp angelehnt an einen realen Use-Case, der beispielhaft als Nachweis der erarbeiteten Bewertung dient.
	

\end{otherlanguage}
