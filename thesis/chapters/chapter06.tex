\chapter{Auswahl relevanter DLTs}
\label{ch:dlt_selection}
In den vorherigen Kapiteln wurden der beispielhafte Anwendungsfall sowie dessen \ac{DLT}-relevante Anforderungen aufgezeigt. In diesem Kapitel wird darauf aufbauend eine Marktübersicht möglicher \ac{DLT}s gegeben und auf die Erfüllung der Anforderungen überprüft.

%
% Section: Vorgehen
%
\section{Vorgehen}
\label{sec:dlt_selection:approach}
Es muss ein geeigentes Vorgehen erarbeitet werden, um aus der unüberschaubaren Menge an Blockchain-Lösungen eine passende Untermenge auszuwählen und dabei keine wichtigen Implementierung zu übersehen. Im Folgenden werden vier mögliche Kriterien vorgestellt, nach denen Blockchain-Lösungen bewertet werden können:
\begin{description}
  \item[Business Relevance] Mitte April 2019 veröffentlichte das Wirtschaftsmagazin Forbes einen Artikel \cite{castillo2019}, der die eingesetzten Blockchain-Technologien großer, internationaler Unternehmen auflistet. Die daraus abzuleitende Relevanz für das internationale Business-Umfeld im Bereich Blockchain stellt einen Indikator dar, der bei der Auswahl einer Blockchain-Lösung berücksichtigt werden muss. Es wurden alle Blockchain-Implementierungen berücksichtigt, die mehr als zwei große Unternehmen betreffen.
  \item[Github Activity] Die Mehrheit der Blockchain-Entwicklungsprojekte sind open-source Projekte; der Quellcode ist meist frei verfügbar und auf Github öffentlich einsehbar. Ein wichtiger Indikator für Auswahl einer \ac{DLT}-Lösung ist die Aktivität der Entwickler auf Github: Es wird bewertet, wie regelmäßig Weiterentwicklungen stattfinden und wie interessiert die Community die Änderungen verfolgen. Je größer die Aktivität und Beliebtheit, desto wahrscheinlicher werden aktuelle Forschungsergebnisse und Neuerungen in die Lösung eingebaut. Um diese Werte vergleichen zu können, wurde sich der Quellen \url{www.coincodecap.de} und \url{www.cryptomismo.de} bedient: Die Anzahl an Commits, aktiven Entwicklern, interessierten Beobachtern und weiteren Kennzahlen wurde gewichtet und bewertet. Das Ergebnis ist ein Ranking der aktivsten Blockchain-Projekte auf Github.
  \item[Blockchain Activity] Die Website \url{www.blocktivity.info} misst die Akivität einer Blockchain-Plattform, indem es die Anzahl an Operationen (Transaktionen, Votes, Blogposts, etc.) verschiedener Zeitpunkte mit der Marktkapitalisierung in Relation setzt. Darüber hinaus werden Angaben über die tatsächlich verwendeten und noch verfügbare Kapazitäten der Lösungen gemacht. Diese Angaben helfen, verschiedene Lösungen gegenüberzustellen und Kapazitäten abzuschätzen. Für die Relevanz im Kontext der Marktübersicht wurden die besten zehn Lösungen hinsichtlich Blockchain-Aktivität berücksichtigt.
  \item[IOT Suitability] Da diese Arbeit von der Synergie zwischen \ac{DLT} und \ac{IOT} handelt, bietet es sich an, das Themenfeld \ac{IOT} bei der Auswahl zu berücksichtigen. Dazu wurden die zehn wertvollsten (nach Marktkapitalisierung) Blockchain-Lösungen, die als \ac{IOT}-Lösung bei der Informationsplattform \url{www.cryptoslate.com} geführt werden, berücksichtigt.
\end{description}
Darüber hinaus fokussiert sich diese Arbeit auf eine Auswahl eigenständiger Blockchains; ERC-Token\footnote{Blockchains basierend auf einem Ethereum-Standard; keine eigenständige Blockchain.} oder ähnliche Implementierungen (Blockchains basierend auf anderen Blockchains) wurden nicht weiter beachtet. Die genauen Daten der einzelnen Kriterien sind im Anhang dieser Arbeit zu finden.

%
% Section: Marktübersicht DLTs
%
\section{Marktübersicht DLTs}
\label{sec:dlt_selection:market}
Nach den im vorherigen Abschnitt vorgestellten Kriterien wurden 30 Blockchain-Lösungen identifiziert. Eine detaillierte Übersicht der Einzelnachweise ist im Anhang zu finden. Die Kriterien wurden folgendermaßen gewichtet: Die Business-Relevanz wird als wichtigstes Kriterium bewertet. Die IOT-Zugehörigkeit wird wichtiger als die Aktualität aber etwas geringer als die Business-Relevanz bewertet. Die Aktualität der Blockchain unterteilt sich in Blockchain-Aktivität und Github-Aktivität; diese werden gleich, allerdings geringer als die IOT-Zugehörigkeit gewichtet.\\
Alle Lösungen werden überprüft, ob sie eine oder mehrere Kiterien erfüllen. Für jedes erfüllte Kriterium wurden der Lösung entsprechend der Gewichtungen Punkte angerechnet. Die Tabelle \ref{tab:dlt_preselection} listet die Erstauswahl gemäß der genannten Kriterien auf.\\
Die Blockchain-Lösung Ethereum führt die Rangliste mit vier Punkten an; das Schlusslicht stellt unter Anderem Zcash mit einem Punkt dar. Um für die weitere Betrachtung von Interesse zu sein, muss eine Lösung zu den Top10 (nach Punkten) gehören; dies entspricht 2 Punkten. Dadurch ergeben sich 11 mögliche Kandidaten (mit mind. 2 Punkten) für die Untersuchung, inwieweit diese die \ac{DLT}-relevanten Anforderungen (vgl. Tabelle \ref{tab:dlt_relevant}) erfüllen.

% Please add the following required packages to your document preamble:
% \usepackage{booktabs}
\begin{table}[]
\begin{tabular}{lccccr}
\hline
\textbf{} & \textbf{\begin{tabular}[c]{@{}c@{}}Business\\ Relevance\end{tabular}} & \textbf{\begin{tabular}[c]{@{}c@{}}Github\\ Activity\end{tabular}} & \textbf{\begin{tabular}[c]{@{}c@{}}Blockchain\\ Activity\end{tabular}} & \textbf{IOT} & \textbf{TOTAL} \\
 & 2 & 1 & 1 & 1,5 &  \\ \hline
Bitcoin & x & x &  &  & 3 \\
Bitcoin SV &  &  & x &  & 1 \\
BitcoinCash & x & x &  &  & 3 \\
Cardano &  & x &  &  & 1 \\
Corda & x &  &  &  & 2 \\
Cosmos &  & x &  &  & 1 \\
EOS &  & x & x &  & 2 \\
Ethereum & x & x & x &  & 4 \\
Hyperledger & x &  &  &  & 2 \\
INT Chain &  &  &  & x & 1,5 \\
IOST &  &  & x &  & 1 \\
IoT Chain &  &  &  & x & 1,5 \\
IOTA &  & x &  & x & 2,5 \\
KIN &  &  & x &  & 1 \\
LBRY Credits &  & x &  &  & 1 \\
Lisk &  & x &  &  & 1 \\
Monero &  & x &  &  & 1 \\
Nano &  &  & x &  & 1 \\
Particl &  & x &  &  & 1 \\
Quorum & x &  &  &  & 2 \\
Ripple & x &  &  &  & 2 \\
Ruff &  &  &  & x & 1,5 \\
SDChain &  &  &  & x & 1,5 \\
Steem &  &  & x &  & 1 \\
Stellar &  & x & x &  & 2 \\
Syscoin &  & x &  &  & 1 \\
Telos &  &  & x &  & 1 \\
Tron &  & x & x &  & 2 \\
WAVES &  & x &  &  & 1 \\
Zcash &  & x &  &  & 1 \\ \hline
\end{tabular}
\caption{Erstauswahl von DLT-Lösungen}
\label{tab:dlt_preselection}
\end{table}

%
% Section: Anforderungserfüllung
%
\section{Anforderungserfüllung}
\label{sec:dlt_selection:fullfilment}
In Kapitel \ref{sec:requirements:transfer} wurden die zurvor als \ac{DLT}-relevant eingestuften Anforderungen in den Kontext \ac{DLT} übersetzt. Dabei ergaben sich sechs Funktionalitäten, die eine Blockchain-Lösung implementieren muss, um für den Einsatz im vorliegenden Anwendungsfall geeignet zu sein. Diese waren Smart-Contracts, Zahlungsmittel, Oracle-Services, Asynchronität, Performanz und Verschlüsselung. Die Tabelle \ref{tab:dlt_detailed_selection} zeigt die 11 verbliebenen Kandidaten und deren Anforderungserfüllung alphabetisch geordnet auf. Dabei wird bewertet, ob eine Anforderung erfüllt ist \textit{[yes]} oder nicht \textit{[no]}. Kann eine Anforderung nur teilweise, sehr schwierig, oder nur unter bestimmten Voraussetzungen erfüllt werden, so wurde diese Anforderung entsprechend mit \textit{[(yes)]} bewertet. Die Informationen der Tabelle wurden den technischen Spezifikationen oder den offiziellen Internetseiten der jeweiligen Lösungen entnommen.

% Please add the following required packages to your document preamble:
% \usepackage{booktabs}
\begin{table}[]
\begin{tabular}{@{}lllllll@{}}
\toprule
\multicolumn{1}{c}{\textbf{}} & \multicolumn{1}{c}{\textbf{\begin{tabular}[c]{@{}c@{}}Smart-\\ Contracts\end{tabular}}} & \multicolumn{1}{c}{\textbf{Payment}} & \multicolumn{1}{c}{\textbf{\begin{tabular}[c]{@{}c@{}}Oracle-\\ Services\end{tabular}}} & \multicolumn{1}{c}{\textbf{\begin{tabular}[c]{@{}c@{}}Perf\\ (TPS)\end{tabular}}} & \multicolumn{1}{c}{\textbf{Async.}} & \multicolumn{1}{c}{\textbf{\begin{tabular}[c]{@{}c@{}}Tx\\ Encrypt.\end{tabular}}} \\ \midrule
Bitcoin & (yes) & yes & (yes) & \textless 10 & yes & no \\
BitcoinCash & (yes) & yes & (yes) & \textless 100 & no & no \\
Corda & yes & (yes) & yes & $\sim$1000 & yes & yes \\
EOS & yes & yes & yes & \textgreater 1000 & yes & no \\
Ethereum & yes & yes & yes & \textgreater 10 & yes & yes \\
Hyperledger & yes & (yes) & yes & - & yes & yes \\
IOTA & no & yes & no & \textgreater 100 & yes & yes \\
Quorum & yes & yes & yes & \textless 1000 & yes & yes \\
Ripple & no & yes & no & \textgreater 1000 & yes & no \\
Stellar & yes & yes & (yes) & \textgreater 1000 & yes & no \\
Tron & yes & yes & yes & \textless 1000 & yes & no \\ \bottomrule
\end{tabular}
\caption{Erfüllung der DLT-relevanten Anforderungen}
\label{tab:dlt_detailed_selection}
\end{table}

Die Performanz bei privaten Blockchains (vgl. Kapitel \ref{subsec:fundamentals:dlt:classification}) wurde ausgeklammert oder nicht angegeben, da diese sehr stark von der zugrundeliegenden Hardware abhängig ist. Darüber hinaus handelt es sich bei den Performanz-Werten um grobe Richtwerte, die teilweise nur unter Laborbedingungen erreicht werden können. Die tatsächlichen Tageswerte weichen zum Teil deutlich davon ab, daher sind diese Werte als Richtwerte zu verstehen und in der folgenden Bewertung als solche zu handhaben.

%
% Section: Bewertung, Ranking & Auswahl
%
\section{Bewertung, Ranking \& Auswahl}
\label{sec:dlt_selection:rating}
Tabelle \ref{tab:dlt_detailed_selection} zeigt mögliche \ac{DLT}-Kandidaten und deren Erfüllung der übersetzten, \ac{DLT}-relevanten Anforderungen auf. Die zentrale Anforderung ist die Ermöglichung von asynchronen Transaktionen bei Verbindungsverlust von Endgeräten. Lediglich Bitcoin-Cash bietet diese Funktionalität nicht, womit es als möglicher Kandidat ausscheidet. Eine weitere, entscheidende Anforderung ist die Bereitstellung von Smart-Contract Funktionalität. Dies ist sowohl bei IOTA als auch bei Ripple nicht gegeben, wodurch beide im weiteren Verlauf nicht mehr betrachtet werden. Die Möglichkeit, Transaktionen oder Smart-Contracts zu verschlüsseln, sodass nur beteiligte Parteien den Inhalt einsehen können, ist bei Bitcoin, Bitcoin-Cash, EOS, Ripple, Stellar und Tron nicht gegeben, weshalb diese im weiteren Verlauf vernachlässigt werden. Übrig bleiben Corda, Ethereum, Hyperledger und Quorum. Corda und Hyperledger bieten beide keine native Währung an; es besteht allerdings die Möglichkeit, beispielsweise mittels Smart-Contracts eine Token-Lösung umzusetzen. Corda, Quorum und Hyperledger sind für den Einsatz in privaten Blockchains vorgesehen, Ethereum kann darüber hinaus auch als Public-Blockchain zum Einsatz kommen. Bezüglich der Performanz steht Ethereum mit etwa 20 Transaktionen pro Sekunde vergleichsweise schlecht dar.\\
Es zeichnet sich ab, dass die vier Lösungen Corda, Hyperledger, Ethereum und Quorum mögliche Kandidaten für eine prototypische Verprobung des vorliegenden Anwendungsfalls sind. Quorum ist in der Lage, sämtliche Anforderungen zu erfüllen. Es handelt sich um eine von der US-Bank JPMorgan entwickelte Blockchain basierend auf Ethereum. Dabei werden neue Ethereum-Releases mit Mechanismen für erhöhte Privätsphäre, alternativen Konsensmechanismen und als private oder permissioned Blockchain bereitgestellt. Hierbei liegt allerdings auch die Problematik, die ebenfalls Corda und Hyperledger betreffen: Die zentrale Trustless-Eigenschaft geht in privaten und zugangsbeschränkten Blockchains verloren und die Transparenz wird veringert. Darüber hinaus werden die Eintrittshürden für die Teilnahme am Blockchain-Netzwerk stark erhöht. Aufgrund dessen und der Tatsache, dass Quorum zum größten Teil aus Ethereum-Quellcode besteht, fällt für die praktische Verprobung in dieser Arbeit die Wahl der Blockchain-Lösung auf Ethereum. Die nicht optimalen Transaktionsraten werden an dieser Stelle vernachlässigt; aktuelle Entwicklungen bezüglich der Skalierungsansätze von Blockchains (vgl. Kapitel \ref{subsec:fundamentals:dlt:scaling}) lassen darauf hoffen, mittelfristig eine deutlich verbesserte Performanz erzielen zu können.
