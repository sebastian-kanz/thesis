\chapter{Einleitung}
\label{ch:intro}
Die Blockchain ist eine Technologie, deren grundlegende Ansätze keineswegs neu sind: Vor über 1500 Jahren suchten die Einwohner der Insel Yap im Westpazifik, östlich der Philippinen, eine Lösung für ihr Zahlungsmittel-Problem. Ihre Währung - Rai - waren großen runde Steintafeln, die teils mehrere Tonnen wogen und als alltägliches Zahlungsmittel ungeeignet waren. Ihre Lösung: Die Dorfältesten merkten sich, wem welcher Rai-Stein gehörte. Fand eine Transaktion statt, so wurde diese den Dorfältesten mitgeteilt. Das Wissen, wem welcher Stein gehörte und wer von wem etwas gekauft hatte, war über mehrere Köpfe verteilt; die dezentrale Informationsspeicherung war geboren. \cite{forbes2018}\\
1991 beschrieben die Forscher Stuart Haber und W. Scott Stornetta die Idee hinter der Blockchain-Technologie \cite{binance2019}, 2008 veröffentlichte Satoshi Nakamoto das Whitepaper zu Bitcoin \cite{nakamoto2009}, dicht gefolgt von Ethereum \cite{buterin2013}, welches 2013 von Vitalik Buterin erstmals vorgestellt wurde.\\
Und wie sieht es heute aus?\\
Mittlerweile ist ein ganzes Ökosystem an Block\-chain- und \ac{DLT}-Lösungen entstanden: Es existieren digitale Vermögenswerte wie Bitcoin, verteilte Anwendungen auf Basis des Blockchain-Protkolls wie \acp{DApp}s auf Ethereum und digitale Zahlungsmittel von fast über 5000 gelisteten Kryptowährungen.\\

%a decentralized and fair internet where users control their own data and markets prosper from network efficiency and security.

%
% Section: Motivation
%
\section{Motivation \& Problemstellung}
\label{sec:intro:motivation}
Das Telekommunikationsunternehmen Cisco prognostiziert, dass bis zum Jahr 2030 mehr als 500 Milliarden mit dem Internet verbundene \ac{IOT}-Geräte in verschiedenen Bereichen unseres alltäglichen Lebens Einzug gehalten haben werden \cite{cisco2016}. Vernetzte Dinge unseres Alltags wie Kühlschränke, Kaffeemaschinen, die automatisierte Supply-Chain aus dem Business-Umfeld oder eine Smart-City sind nur einige wenige Beispiele dieses Geschäftsfeldes. Obwohl das Konzept von \ac{IOT} noch sehr theoretisch ist, wurden bereits einige Anwendungsfälle erarbeitet. Um das große Potential von \ac{IOT} vollumfänglich nutzbar zu machen und entsprechende Visionen umzusetzen, muss eine passende IT-Lösung für den entsprechenden Anwendungsfall bereitgestellt werden. Viele verschiedene Hersteller und Service-Provider benötigen eine einheitliche Plattform, auf der sie ihre \ac{IOT}-Geräte, Services, Geschäftslogiken und Kunden miteinander vernetzen und ein sicheres Bezahlsystem integrieren können. Es stellt sich die Frage, ob und inwiefern die zwei innovativen Technologien \ac{DLT} und \ac{IOT} voneinander profitieren können und ob sich \ac{DLT} als skalierende, performante und sichere Technologie für \ac{IOT}-Anwendungsfälle eignet.

%
% Section: Zielsetzung \& Zielgruppe
%
\section{Zielsetzung \& Zielgruppe}
\label{sec:intro:goal}
Das Ziel dieser Arbeit ist die Untersuchung und prototypische Verprobung der folgenden These:
\begin{quote}
  '\ac{DLT} eignet sich als Technologie für \ac{IOT} und die nicht-funktionalen Anforderungen sind für alle \ac{DLT}-\ac{IOT}-Anwendungsfälle gleich.'
\end{quote}
Es wird gezeigt, inwieweit sich die Technologie \ac{DLT} für \ac{IOT}-Anwendungsfälle eignet, welche Anforderungen dafür erfüllt sein müssen und welche Implementierung für die Umsetzung in Frage kommt. Dazu wird im Verlauf der Arbeit das zu lösende Problem genauer spezifiziert und herausgearbeitet: Nachdem ein konkreter, stellvertretender \ac{IOT}-Anwendungsfall vorgestellt wurde und alle \ac{DLT}-relevanten Anforderungen ermittelt und evaluiert sind, ergibt sich das konkrete Problem als die Umsetzung eines speziellen \ac{IOT}-Anwendungsfalls mit einer ausgewählten \ac{DLT}-Lösung. Der explizite Lösungsraum, also welche Anforderungen umgesetzt werden, wird zuvor genau beschrieben. Das Problem gilt als gelöst, sobald die zuvor abgeleiteten Anforderungen mit dem \ac{PoC} erfüllt werden können.\\
Abschließend wird die eingangs formulierte These diskutiert und evaluiert.\\
Diese Arbeit richtet sich an IT-Spezialisten aus dem Umfeld \ac{DLT} und \ac{IOT}, die sich über die Synergie beider Konzepte informieren, sowie Fachleute aus der Industrie, die entsprechende \ac{IOT}-Anwendungsfälle ausarbeiten möchten. Ein solides Grundverständnis für die grundlegenden Konzepte und Wordings wird an dieser Stelle vorausgesetzt; auf entsprechende Grundlagenliteratur wird gegebenenfalls verwiesen.

%
% Section: Methodik & Aufbau dieser Arbeit
%
\section{Methodik \& Aufbau dieser Arbeit}
\label{sec:intro:methodology}
In dieser Arbeit werden die Themenbereiche '\ac{DLT}' und '\ac{IOT}' vorgestellt, klassifiziert und '\ac{DLT}' als Kommunikationsprotokoll eingeordnet. Die Synergie beider Bereiche wird herausgearbeitet und es wird dem Leser vorgestellt, wie diese Technologien voneinander profitieren können (Kapitel \ref{ch:fundamentals}).\\
Zur entsprechenden Einordnung dieser Arbeit werden verwandte Forschungsarbeiten vorgestellt (Kapitel \ref{ch:relatedwork}). Ein beispielhafter \ac{IOT}-Anwendungsfall wird entwickelt (Kapitel \ref{ch:iot_usecase}) und eine detaillierte Auflistung aller Anforderungen erarbeitet (Kapitel \ref{ch:requirements}).\\
Im nächsten Schritt werden die ermittelten Anforderungen schrittweise auf eine Untermenge von fundamentalen Anforderungen reduziert, die relevant für \ac{IOT} in Verbindung mit \ac{DLT} sind. Mehrere, am Markt etablierte \ac{DLT}s werden anschließend vorgestellt und auf Basis dieser Anforderungsuntermenge evaluiert (Kapitel \ref{ch:dlt_selection}). Es wird geprüft, ob und inwieweit sie sich als Lösung für den \ac{IOT}-Anwendungsfall qualifizieren. Die vielversprechendste Lösung wird in einem \ac{PoC} prototypisch umgesetzt (Kapitel \ref{ch:implementation}), um die Anforderungsliste zu evaluieren. Es wird gezeigt, dass die gewählte \ac{DLT} zielbringend eingesetzt werden kann (Kapitel \ref{ch:results}).\\
Anschließend wird diskutiert, ob und warum die nicht-funktionalen Anforderungen für \ac{DLT}-geeignete \ac{IOT}-Anwendungsfälle - unabhängig vom tatsächlichen Anwendungsfall selbst - stets die gleichen sind (Kapitel \ref{ch:discussion}). Abschließend wird ein Ausblick über weitere, mögliche Forschungsgebiete in diesem Umfeld gegeben (Kapitel \ref{ch:perspective}).\\

Diese Arbeit zeigt die Eignung von verschiedenen \ac{DLT}s für \ac{IOT}-Anwendungsfälle anhand eines beispielhaften \ac{PoC}s. Es werden nur solche Bereiche von \ac{IOT} betrachtet, die auch grundsätzlich für die Implementierung auf \ac{DLT}s geeignet sind. Es gibt darüber hinaus weitere Bereiche, die sich nicht eignen, auf \ac{DLT}s umgesetzt zu werden und diese müssen auf einer anderen technologischen Basis implementiert werden (vgl. Kapitel \ref{sec:fundamentals:iot}). Die in dieser Arbeit durchgeführte Analyse wird anhand eines \ac{PoC} belegt. Aufgrund von Restriktionen bezüglich Zeitlimitierung und praktischer Umsetzbarkeit könnten unter Umständen nicht alle fundamentalen Anforderungen gezeigt werden, die für einen \ac{IOT}-\ac{DLT}-Anwendungsfall erfüllt sein müssen (vgl. Kapitel \ref{ch:implementation}).
