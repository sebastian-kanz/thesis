\chapter{Einleitung}
\label{ch:intro}
Geschichtlicher Verlauf: Merkle-Tree, Blockchain, Bitcoin, Ethereum
Heute: Smart-Contracts und \ac{DApp}s
Die Technologie
\footnote{Binance Academy: https://www.binance.vision/de/blockchain/history-of-blockchain}

%
% Section: Motivation
%
\section{Motivation \& Problemstellung}
\label{sec:intro:motivation}
Der Begriff Blockchain - oder allgemeiner \ac{DLT} - wird heute oftmals synonym für Bitcoin oder dessen Artgenossen verwendet. Dabei handelt es sich um eine Technologie, die bereits mehr und mehr Einzug in unseren Alltag erhält: Sei es ein digitales Zahlungsmittel wie Bitcoin, ein global verteilter Supercomputer wie Golem oder ein komplexes System wie Ethereum, auf welchem sich umfangreiche Geschäftslogiken mittels Smart-Contracts umsetzen lassen; die Möglichkeiten und Anwendungsfälle scheinen schier unbegrenzt zu sein. Es entstehen fast täglich neue Anwendungsfälle; unter Anderem auch im Umfeld der noch recht jungen Technik hinter \ac{IOT}.\\
Das Telekommunikationsunternehmen Cisco progostiziert, dass bis zum Jahr 2030 mehr als 500 Milliarden mit dem Internet verbundene \ac{IOT}-Geräte in verschiedenen Bereichen unseres alltäglichen Lebens Einzug erhalten haben werden\footnote{https://www.cisco.com/c/dam/en/us/products/collateral/se/internet-of-things/at-a-glance-c45-731471.pdf}. Vernetzte Dinge unseres Alltags wie Kühlschranke oder Kaffeemaschinen, aber auch die aus dem Business-Umfeld autmatisierte Supply-Chain oder eine Smart-City sind nur einige wenige Beispiele dieses Geschäftsfeldes. Das Konzept von \ac{IOT} ist nach wie vor sehr theoretisch, obwohl bereits einige Anwendungsfälle erarbeitet wurden. Um das große Potential von \ac{IOT} vollumfänglich nutzbar zu machen und entsprechende Visionen umzusetzen, muss eine passende Backbone-Lösung für solche Anwendungsfälle bereitgestellt werden. Viele verschiedene Hersteller und Service-Provider benötigen eine einheitliche Plattform, auf der sie ihre \ac{IOT}-Geräte, Services, Geschäftslogiken und Kunden miteinander vernetzen können sowie die Integration eines sicheren Bezahlsystems. Es stellt sich die Frage, ob und inwiefern diese zwei innovativen Technologien voneinader profitieren können und ob \ac{DLT} als hoch-skalierende, performante und sichere Backbone-Technologie für \ac{IOT}-Anwendungsfälle eignet.

%
% Section: Zielsetzung \& Zielgruppe
%
\section{Zielsetzung \& Zielgruppe}
\label{sec:intro:goal}
Das Ziel dieser Arbeit ist zum einen die Untersuchung der folgenden These: '\ac{DLT} eignet sich als Backbone-Technologie für \ac{IOT} und die technischen / nicht-funktionalen Anforderungen sind für alle Anwendungsfälle gleich'. Es wird gezeigt, inwieweit sich die Technologie \ac{DLT} als Backbone-System für \ac{IOT}-Anwendungsfälle eignet, welche Anforderungen dafür erfüllt sein müssen, und welche Implementierung für die Umsetzung in Frage kommt. Zum anderen wird im Verlauf der Arbeit das zu lösende Problem genauer spezifiziert und herausgearbeitet. Nachdem ein konkreter Anwendungsfall vorgestellt wurde und alle \ac{DLT}-relevanten Anforderungen ermittelt und evaluiert sind, ergibt sich ein Problem der Form 'Ich möchte den \ac{IOT}-Anwendungsfall [Name] mit \ac{DLT} [Name] lösen', wobei der Lösungsraum, also welche Anforderungen umgesetzt werden sollen, zuvor genau beschrieben wurde. Das Problem gilt als gelöst, sobald die zuvor abgeleiteten Anforderungen mit dem \ac{PoC} erfüllt werden können. Abschließend wird die eingangs formulierte These diskutiert und evaluiert.\\
Diese Arbeit richtet sich an IT-Spezialisten aus dem Umfeld \ac{DLT} und \ac{IOT}, die sich über die Synergie beider Konzepte informieren, sowie Fachleuten aus der Industrie, die entsprechende \ac{IOT}-Anwendungsfälle ausarbeiten möchten. Ein solides Grundverständnis für die grundlegenden Konzepte und Wordings wird an dieser Stelle vorausgesetzt; auf entsprechende Grundlagenliteratur wird an entsprechender Stelle verwiesen.

%
% Section: Methodik
%
\section{Methodik \& Vorgehen}
\label{sec:intro:methodology}
In dieser Arbeit werden die Themenbereiche '\ac{DLT}' und '\ac{IOT}' vorgestellt, klassifiziert und in das \ac{OSI} Referenzmodell eingeordnet. Die Synergie beider Bereiche wird herausgearbeitet und es wird dem Leser vorgestellt, wie diese Technologien voneinander profitieren können. Ein beispielhafter \ac{IOT}-Anwendungsfall wird entwickelt und eine detaillierte Auflistung aller Anforderungen erarbeitet. Im nächsten Schritt werden die die ermittelten Anforderungen schrittweise auf eine Untermenge von fundamentalen Anforderungen reduziert, die relevant für \ac{IOT} in Verbindung mit \ac{DLT} sind. Mehrere, am Markt etablierte \ac{DLT} Anwendungen werden anschließend vorgestellt und auf Basis dieser Untermenge evaluiert. Es wird geprüft, ob und inwieweit sie sich als Backbone-Lösung für den \ac{IOT}-Anwendungsfall qualifizieren. Die vielversprechendste Lösung wird in einem \ac{PoC} umgesetzt, um die Anforderungsliste zu evaluieren. Es wird gezeigt, dass die gewählte \ac{DLT} zielbringend als \ac{IOT} Backbone-Lösung eingesetzt werden kann. Abschließend wird gezeigt, dass die nicht-funktionalen Anforderungen für \ac{DLT}-geeignete \ac{IOT}-Anwendungsfälle, unabhängig vom tatsächlichen Anwendungsfall selbst, stets die gleichen sind.

Diese Arbeit zeigt die Eignung von verschiedenen \ac{DLT}s als Backbone-Lösung für \ac{IOT}-Anwendungsfälle anhand eines beispielhaften \ac{PoC}s. Es werden nur solche Bereiche von \ac{IOT} betrachtet, die auch grundsätzlich für die Implementierung auf \ac{DLT}s geeignet sind. Es gibt darüber hinaus weitere Bereiche, die sich nicht eignen, um auf \ac{DLT}s umgesetzt zu werden und müssen auf einer anderen technologischen Basis implementiert werden. Des weiteren wird die in dieser Arbeit durchgeführte Analyse anhand eines \ac{PoC} belegt. Aufgrund von Restriktionen wie der Zeitlimitierung und der praktischen Umsetzbarkeit könnten unter Umständen nicht alle fundamentalen Anforderungen gezeigt werden, die für einen \ac{IOT}-\ac{DLT}-Anwendungsfall erfüllt sein müssen.


%
% Section: Aufbau dieser Arbeit
%
\section{Aufbau dieser Arbeit}
\label{sec:intro:structure}
\lipsum[1-1]
