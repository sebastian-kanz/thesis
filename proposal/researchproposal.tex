\documentclass[
    ngerman,american
    ]{scrartcl}

    % ##########################################
    % # Choose the language for the document by editing below line
    % # de = German
    % # en = English
    \newcommand{\lang}{en}
    % ##########################################

    \usepackage{babel}
    \usepackage[utf8]{inputenc}
    \usepackage{csquotes}
    \usepackage{enumitem}
    \usepackage{ifthen}
    \usepackage{lipsum}

    \newcommand{\paperSubTitle}[1]
{
    \ifthenelse{\equal{#1}{en}}{Outline and Topic Proposal}{}
    \ifthenelse{\equal{#1}{de}}{Outline und Themenvorschlag}{}
}

\newcommand{\sectionQuestions}[1]
{
    \ifthenelse{\equal{#1}{en}}{\section{Scope of Work - 4 Questions}}{}
    \ifthenelse{\equal{#1}{de}}{\section{Ziel der Arbeit - 4 Fragen}}{}
}

\newcommand{\sectionQuestionsDescription}[1]
{
    \ifthenelse{\equal{#1}{en}}{In this section the essence of the proposed work is described by answering four key questions. }{}
    \ifthenelse{\equal{#1}{de}}{Im Folgenden wird der Kern der Arbeit beschrieben indem vier Kernfragen beantwortet werden.}{}
}

\newcommand{\sectionThesis}[1]
{
    \ifthenelse{\equal{#1}{en}}{\section{Thesis}}{}
    \ifthenelse{\equal{#1}{de}}{\section{These}}{}
}

\newcommand{\sectionThesisDescription}[1]
{
    \ifthenelse{\equal{#1}{en}}{The thesis analyzed in this work is as follows:}{}
    \ifthenelse{\equal{#1}{de}}{Die These, die in dieser Arbeit untersucht wird, lautet wie folgt:}{}
}

\newcommand{\sectionInitialTOC}[1]
{
    \ifthenelse{\equal{#1}{en}}{\section{Preliminary Table of Contents}}{}
    \ifthenelse{\equal{#1}{de}}{\section{Vorläufige Gliederung}}{}
}

\newcommand{\sectionInitialTOCDescription}[1]
{
    \ifthenelse{\equal{#1}{en}}{In this section the table of contents for the proposed work is described.}{}
    \ifthenelse{\equal{#1}{de}}{Im Folgenden wird ein Inhaltverzeichnis für die vorgeschlagene Arbeit vorgestellt.}{}
}

\newcommand{\sectionSource}[1]
{
    \ifthenelse{\equal{#1}{en}}{\section{Relevant Related Work}}{}
    \ifthenelse{\equal{#1}{de}}{\section{Relevante verwandte Arbeiten}}{}
}


\newcommand{\sectionSourceDescription}[1]
{
    \ifthenelse{\equal{#1}{en}}{In this section, identified related work is described.}{}
    \ifthenelse{\equal{#1}{de}}{Diese Section stellt verwandte Arbeiten dar und erklärt kurz deren Bedeutung für die vorgeschlagene Arbeit.}{}
}

\newcommand{\questionOne}[1]
{
    \ifthenelse{\equal{#1}{en}}{What is the problem you want to address in your work?}{}
    \ifthenelse{\equal{#1}{de}}{Was ist das Problem, welches Sie in Ihrer Arbeit bearbeiten wollen?}{}
}

\newcommand{\questionTwo}[1]
{
    \ifthenelse{\equal{#1}{en}}{Why is it a problem?}{}
    \ifthenelse{\equal{#1}{de}}{Warum ist es ein Problem?}{}
}

\newcommand{\questionThree}[1]
{
    \ifthenelse{\equal{#1}{en}}{What is the solution you developed in your work?}{}
    \ifthenelse{\equal{#1}{de}}{Was ist die Lösung die sie entwickelt haben?}{}
}

\newcommand{\questionFour}[1]
{
    \ifthenelse{\equal{#1}{en}}{Why is it a solution?}{}
    \ifthenelse{\equal{#1}{de}}{Warum ist es eine Lösung?}{}
}


    \ifthenelse{\equal{en}{\lang}}
    {
        \selectlanguage{american}
    }{
        \ifthenelse{\equal{de}{\lang}}
        {
            \selectlanguage{ngerman}
        }
        {\selectlanguage{american}}
    }

    \usepackage[
        bibencoding=utf8,
        style=alphabetic
    ]{biblatex}

    \bibliography{bibliography}


    \usepackage{amsmath}
    \title{
        % ##########################################
        % # Insert the title of your paper/thesis here
        % ######
        % Coming up with a good title is hard.
        % It should:
        %  1. capture the contents of the your work
        %  2. not be to broad or generic
        %  3. stick to the truth and don't not oversell
        %  4. use established terms and wordings
        %  5. make people curious about your work
        %  6. use current buzzwords if possilbe (but do it right)
        %  7. not use too many buzzwords :-)
        Distributed ledger technologies as backbone solution for IOT applications
        % ##########################################
        \\  \Large{\paperSubTitle{\lang}}} % don't touch this line

    \author{
        % ##########################################
        % # Your name goes here
        % ######
        % wWll, that should be obvious, right?
        Sebastian Kanz
        % ##########################################
        }

    \begin{document}
      \maketitle
        \begin{abstract}
            % ##########################################
            % # Include your Abstract here
            % ######
            % I would strongly suggest to start working on the abstract only
            % after you have answered the 4 questions in Section 1, as this will
            % make it much easier for you to come up with an abstract that
            % is to the point, short, and still summarizing all the most crucial
            % results of your work.
            %
            % The abstract should include the following points:
            %  - a short but to the point introduction of the problem area
            %  - what is the topic/problem, tackled in your work?
            %  - why is the topic/problem of your work relevant? Why should the
            %    reader care about it?
            %  - what are the results/answers of your work?
            %  - how did you gain your results and what is their quality?
            %                %
            % It should NOT be:
            %  - too long/verbose
            %  - too short
            Das Internet der Dinge (engl. IOT; Internet of Things) erhält immer mehr Einzug in das tägliche Leben. Smart-Home Lösungen, vernetzte und latenzempfindliche Connected-Cars oder die Vision einer Smart-City prägen die Forschungsarbeiten in den jeweiligen Bereichen. Ziel ist eine vollautomatische Machine-to-Machine (M2M) Abwicklung von Prozessen, um unseren Alltag zu automatisieren und zu vereinfachen. Dabei fallen eine Menge Daten an, die verarbeitet, übertragen und gespeichert werden müssen. Mit der Einführung des neuen Mobilfunk-Standards 5G und immer leistungsfähigeren Endgeräten sind Übertragung und Verarbeitung der Daten weitestgehend gesichert; bleibt die Frage offen, wo diese großen Datenmengen gespeichert und weiterprozessiert werden.
            Daneben erfreut sich das Thema Distributed-Ledger-Technology (DLT) immer größerer Beliebtheit: Es werden täglich neue Anwendungsfälle gefunden, die durch die verteilte Infrastruktur, der Trustless-Eigenschaft und der Dezentralität profitieren. Es bietet sich eine Untersuchung an, um zu überprüfen, inwieweit diese beiden noch recht jungen Technologien IOT und DLT Synergien besitzen und sich gegebenenfalls gegenseitig ergänzen können.
            Die vorliegende Masterarbeit evaluiert eine Auswahl etablierter DLTs anhand ihrer Tauglichkeit für den Einsatz im IOT-Umfeld mit Fokus auf den M2M-Bereich. Dazu wird zunächst ein IOT-Anwendungsfall erstellt, der stellvertretend für den M2M-Bereich für die weiteren Analysen verwendet wird. Anschließend werden konkrete Anforderungen aus verschiedenen Bereichen Infrastruktur, IT-Security, Performance und weiteren aufgestellt, die eine DLT erfüllen muss, um den Anforderungen des beispielhaften Anwendungsfalls gerecht zu werden. Die erstellten Kriterien werden auf eine Auswahl von DLT-Implementierungen angewandt, evaluiert und bewertet. Mit der am besten geeigneten DLT wird eine prototypische Implementierung des Anwendungsfalls vorgenommen, um die Ergebnisse aus der Anforderungsevaluierung zu überprüfen. Um den Use-Case möglichst realistisch zu simulieren werden Daten aus verschiedenen IOT-Sensoren an die DLT übermittelt und eine M2M-Komminkation zwischen IOT-Devices via DLT erstellt. Anschließende Load-Tests geben detaillierte Informationen über die Performance.
            Das Ergebnis ist eine strukturierte und nachvollziehbare Bewertung mehrerer, am Markt etablierter DLTs, inwieweit diese für DLT-sinnvolle IOT-Anwendungsfälle im M2M-Umfeld geeignet sind, sowie ein DLT-basierter Prototyp angelehnt an einen realen Use-Case, der beispielhaft als Nachweis der erarbeiteten Bewertung dient.
            % ##########################################

        \end{abstract}

        \newpage

        \sectionThesis{\lang}
        \sectionThesisDescription{\lang}
                \begin{center}
                  'DLTs are suitable for beeing an IOT backbone solution and the non-functional requirements are independant from the actual usecase.'
                \end{center}
        \sectionQuestions{\lang}
        \sectionQuestionsDescription{\lang}

        \begin{description}[style=unboxed]
            \item [\questionOne{\lang}]
                % ##########################################
                % # Question 1: What is the problem you want to address in your work? /
                %               Was ist das Problem, welches Sie in Ihrer Arbeit bearbeiten wollen?
                % ######
                % The goal of this question is to clearly state what your work is about.
                % What is the problem it is supposed to solve?
                %
                % Answering this question is particular important during the early phases
                % of your work, in order to gain further insight and understanding about what
                % your work is going to cover and address.
                %
                % Answer this question very briefly by stating the problem or research
                % question that you want to address/solve in your work.
                %
                % Your answer should:
                %  - only be 1 sentence (2 sentences max)
                %  - not cover a statement why the topic is relevant
                %    for the industry (this is address by the next question)
                %  - properly use common terms and buzzwords of IT today (similar to the
                %    rules for the abstract)
                %
                % Please acknowledge: the answer to this question should not cover why the
                % problem is important or relevant to anyone (e.g. industry). This will
                % be addressed with the next question.
                There is no suitable solution for a backbone system for large-scaling, performant and secure IOT usecases where multiple distrusting parties are involved.

                % ##########################################

            \item [\questionTwo{\lang}]
                % ##########################################
                % # Question 2: Why is it a problem? / Warum ist es ein Problem?
                % ######
                % The goal of this question is to describe why your work is relevant.
                % Why should the reader care? Why is this the problem (of question 1)
                % worth investigating?
                %
                % Answering this question is particular important during the early phases
                % of your work, in order to gain further insight and understanding of the
                % problem domain you are addressing. Further, it is a good checkpoint to
                % ensure that you are addressing issues that are not just theoretical but
                % have real-world applications.
                %
                % Your answer should:
                %  - be 3 - 5 sentences
                %  - give a broader overview of the domain/area where your problem occurs
                %  -- who has this problem?
                %  -- what is the impact of it?
                %  -- which conditions need to be fulfilled for the problem to occur?
                %  -- etc.
                %  - describe the benefit of having the problem resolved
                According to Cisco there will be more than 500 billion IOT devices connected to the internet by 2030 in many areas of our everyday lives \footnote{https://www.cisco.com/c/dam/en/us/products/collateral/se/internet-of-things/at-a-glance-c45-731471.pdf}.
                The concept of IOT is still mostly theoretical although some usecases are already developed.
                To accomplish the full potential of the IOT concept a suitable backbone solution must be provided.
                Different vendors and service providers need a common plattform for connecting their devices, services, buiness logic as well as their customers and establish a suitable payment solution.

                % ##########################################

            \item [\questionThree{\lang}]
                % ##########################################
                % # Question 3: What is the solution you developed in your work? /
                %               Was ist die Lösung die sie entwickelt haben?
                % ######
                % The goal of this question is to describe the results of your work
                % ans/or solution to the problem of your work.
                %
                % It is hard/impossible to answer this question in the early phases of
                % your work, as usually you do not have results, yet. However, you can
                % already state first ideas that you may have in order to discuss them
                % with your supervisor.
                %
                % Your answer should:
                %  - clearly state all results of your work, that are relevant to your
                %    research problem.
                %  - not oversell your results, stick you what you actually have
                %    accomplished"
                %  - give credit where credit is due. If you created your results based
                %    on the work of others, give them the credit.
                %  - if none of your ideas did not produce any usable solution, state
                %    so - these attempts are also results! By documenting them, it may
                %    prevent others from trying them as well.
                In this work the common topics 'IOT' and 'DLT' are presented and classified in the OSI-model. The intersection of both topics is shown and how they might benefit from each other.
                An exemplary IOT usecase is developed and a detailed list of all requirements is defined. In a next step those requirements are recursivley reduced to a subset of fundamental requirements that are relevant for IOT in conjunction with DLTs.
                Different well-established DLT-solutions are presented and evaluated against the subset of requirements to show if and how they qualify to be a backbone solution for that usecase.
                The best fitting DLT-solution is implemented as PoC to evaluate the earlier highlighted requirements. It is shown, that the specific DLT-solution is suitable for an IOT-usecase and can serve as a backbone solution.
                Furthermore it is pointed out that the non-functional requirements for DLT-suitable IOT-usecases are identical - independant from the IOT-usecase itself.

                % ##########################################

            \item [\questionFour{\lang}]
                % ##########################################
                % # Question 4: Why is it a solution? / Warum ist es eine Lösung?
                % ######
                % The goal of this question is to describe who you developed your results
                % and what the quality of them are.
                %
                % Your answer should:
                %  - short and to the point
                %  - clearly state how you developed your results
                %  -- what is the chain of reasoning that led to your results/solution
                %  -- what statistics, literature, studies, or other literature did you
                %     base your assumptions on?
                %  - clearly state the quality and applicability of your results
                %  -- reflect your work objectively - there is no perfection in this
                %     world, so your work is not perfect as well. Be aware of that!
                %  -- how did you ensure that your results are accurate? did you:
                %  --- perform experiments?
                %  --- apply any logical deductions?
                %  --- mathematical proofs?
                %  --- implement a "proof of concept" implementation and evaluate?
                %  --- etc.
                %  - clearly state the shortcomings of your work
                %  -- be hones and objective about your own work.
                %  -- In which cases/scenarios are your results applicable?
                This work proves the usability of DLTs as IOT backbone solutions with an exemplary PoC.
                Only those parts of IOT that are suitable for being implemented on DLTs are analyzed. There are many more IOT patterns that are not fitting with DLT and might be implemented with other technologies.
                Furthermore the analysis drawn in this work is evaluated with a PoC implementation. Due to restrictions like practicability or time limit not all fundamental requirements needed for an IOT backbone solution could be proven by the PoC.


                % ##########################################
        \end{description}

        \newpage

        \sectionInitialTOC{\lang}
        \sectionInitialTOCDescription{\lang}

        % ##########################################
        % # Proposed table of contents
        % ######
        % The goal of this section is to propose a table of contents. Please keep in
        % mind: a well created table of contents is very powerful in provide a good
        % overview of the overall chain of reasoning of your work. This makes it
        % extremely valuable.
        %
        % Please include:
        %  - names of sections and subsections (please don't go deeper than that unless
        %    your supervisor asks you for it)
        %  - a brief description of the proposed content of each section and subsection
        %    (1-3 sentences)
        %
        \begin{enumerate}

            \item \textbf{Section 1 Einleitung}
                    \begin{enumerate}
                        \item \textbf{Subsection 1 Motivation}
                        \item \textbf{Subsection 2 Problemstellung}
                        \item \textbf{Subsection 3 Zielsetzung}
                        \item \textbf{Subsection 4 Zielgruppe}
                        \item \textbf{Subsection 5 Verwandte Arbeiten}
                        \item \textbf{Subsection 6 Aufbau dieser Arbeit}
                    \end{enumerate}
            \item \textbf{Section 2 Theoretische Grundlagen}
                    \begin{enumerate}
                        \item \textbf{Subsection 1 Einleitung DLT \& IOT (Referenzen)}
                        \item \textbf{Subsection 2 Defintionen}
                        \item \textbf{Subsection 3 Abgrenzung DLT}
                        \item \textbf{Subsection 4 Abgrenzung IOT}
                        \item \textbf{Subsection 5 Schnittmenge DLT \& IOT} (Wann eignet sich DLT für IOT und wann IOT für DLT?)
                        \item \textbf{Subsection 6 OSI-Referenzmodell}
                        \item \textbf{Subsection 7 Einordung DLT \& IOT ins OSI-Modell} (Blockchain als Kommunikationsprotokoll?)
                    \end{enumerate}
            \item \textbf{Section 3 ["Name des IOT-Anwendungsfalls"]} (Exemplarischer IOT-Anwendungsfall)
                    \begin{enumerate}
                        \item \textbf{Subsection 1 Beschreibung}
                        \item \textbf{Subsection 2 Zielgruppe}
                        \item \textbf{Subsection 3 Architektur / Aufbau}
                        \item \textbf{Subsection 4 Einordnung IOT-Umfeld}
                        \item \textbf{Subsection 5 ...}
                    \end{enumerate}
            \item \textbf{Section 4 Anforderungsanalyse und -evaluierung}
                    \begin{enumerate}
                        \item \textbf{Subsection 1 Anforderungsanalyse IOT-Anwendungsfall}
                        \item \textbf{Subsection 2 Anforderungsklassifizierung} (Funktionale, Nicht-Funktionale, ...)
                        \item \textbf{Subsection 3 Anforderungsevaluierung} (Rekursive Reduktion des Untersuchungsraumes / Relevanz DLT-IOT)
                    \end{enumerate}
            \item \textbf{Section 5 Marktübersicht relevanter DLTs}
                    \begin{enumerate}
                        \item \textbf{Subsection 1 Übersicht}
                        \item \textbf{Subsection 2 Anforderungserfüllung}
                        \item \textbf{Subsection 3 Bewertung / Ranking}
                    \end{enumerate}
            \item \textbf{Section 6 Proof-of-Concept}
                    \begin{enumerate}
                        \item \textbf{Subsection 1 Eingesetzte Technologien}
                        \item \textbf{Subsection 2 Zu untersuchende Anforderungen}
                        \item \textbf{Subsection 3 Architektur}
                        \item \textbf{Subsection 4 Tests}
                    \end{enumerate}
            \item \textbf{Section 7 Diskussion}
                    \begin{enumerate}
                        \item \textbf{Subsection 1 DLT: Eignung als IOT-Backbone?}
                        \item \textbf{Subsection 2 DLT-IOT-Usecase: Technische Anforderungen immer gleich?}
                    \end{enumerate}
            \item \textbf{Section 8 Fazit}
            \item \textbf{Section 9 Ausblick}
                    \begin{enumerate}
                        \item \textbf{Mehrwert gegenüber Cloud / Client-Server / ... ?}
                    \end{enumerate}


        \end{enumerate}
        % ##########################################

        \newpage

        \sectionSource{\lang}
        \sectionSourceDescription{\lang}

        % ##########################################
        % # Overview of identified relevant work
        % ######
        % The goal of this section is to provide an overview of the relevant and significant
        % related work identified so far. Make sure that your cited sources are of appropriate
        % quality!
        %
        % Please include:
        %  - a citation of the source using Latex facilities (incl. a generated list of
        %    references)
        %  - a brief descriptions of the source and a statement why this is relevant for
        %    your work (1-2 sentences)
        %
        \begin{description}
          \item[\cite{DBLP:journals/corr/abs-1809-05613}] The authors analyze many blockchain consensus algorithms for their suitability for IOT solutions. The Tangle-consensus algorithm of IOTA seems to fit best as it utilizes a DAG.
          \item[\cite{iotbds18}] The authors present a decision framework for when to use blockchain technology and when not.
          \item[\cite{8632193}] The authors analyze DAGs (namely Nxt, IOTA, Orumesh, DagCoin,Byteball, Nano and XDAG, even though some of them are only ERC20-Token) and their advantages over classical blockchain solutions. They do not describe a specific usecase or a concrete application.
          \item[work in progress]
        \end{description}
        % ##########################################

        \newpage

      \printbibliography
    \end{document}
