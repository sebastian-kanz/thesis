%*******************************************************
% Abstract in German
%*******************************************************
\begin{otherlanguage}{ngerman}
	\pdfbookmark[1]{Zusammenfassung}{Zusammenfassung}
	\chapter*{Zusammenfassung}
	Das Internet der Dinge (engl. IOT; Internet of Things) erhält immer mehr Einzug in das tägliche Leben. Smart-Home Lösungen, vernetzte und latenzempfindliche Connected-Cars oder die Vision einer Smart-City prägen die Forschungsarbeiten in den jeweiligen Bereichen. Ziel ist eine vollautomatische Machine-to-Machine (M2M) Abwicklung von Prozessen, um unseren Alltag zu automatisieren und zu vereinfachen. Dabei fallen eine Menge Daten an, die verarbeitet, übertragen und gespeichert werden müssen. Mit der Einführung des neuen Mobilfunk-Standards 5G und immer leistungsfähigeren Endgeräten sind Übertragung und Verarbeitung der Daten weitestgehend gesichert; bleibt die Frage offen, wo diese großen Datenmengen gespeichert und weiterprozessiert werden.
	Daneben erfreut sich das Thema Distributed-Ledger-Technology (DLT) immer größerer Beliebtheit: Es werden täglich neue Anwendungsfälle gefunden, die durch die verteilte Infrastruktur, der Trustless-Eigenschaft und der Dezentralität profitieren. Es bietet sich eine Untersuchung an, um zu überprüfen, inwieweit diese beiden noch recht jungen Technologien IOT und DLT Synergien besitzen und sich gegebenenfalls gegenseitig ergänzen können.\\
	Die vorliegende Masterarbeit evaluiert eine Auswahl etablierter DLTs anhand ihrer Tauglichkeit für den Einsatz im IOT-Umfeld mit Fokus auf den M2M-Bereich. Der zentrale Forschungsgegenstand ist die These, dass DLT eine geeignet Technologie für IOT darstellt und dass alle DLT-IOT-Anwendungsfälle die gleichen nicht-funktionalen Anforderungen besitzen. Dazu wird zunächst ein IOT-Anwendungsfall erstellt, der für die weiteren Analysen verwendet wird. Anschließend werden konkrete Anforderungen mit Hilfe eines Klassifizierungsmodells aufgestellt, die eine DLT erfüllen muss, um dem Anwendungsfall gerecht zu werden. Die erstellten Kriterien werden auf eine Auswahl von DLT-Implementierungen angewandt, evaluiert und bewertet. In der Analyse stellte sich die Blockchain-Plattform Ethereum als geeignetste Lösung heraus. Mit dieser wurde eine prototypische Implementierung des Anwendungsfalls vorgenommen, um das Resultat der Anforderungsevaluierung und letztlich auch die Forschungsfrage zu überprüfen.\\
	Das Ergebnis ist eine strukturierte und nachvollziehbare Bewertung mehrerer, am Markt etablierter DLTs, inwieweit diese für DLT-sinnvolle IOT-Anwendungsfälle geeignet sind, sowie ein DLT-basierter Prototyp, angelehnt an einen realen Use-Case, der beispielhaft als Nachweis der erarbeiteten Bewertung dient. Es wird gezeigt, dass IOT unter gewissen Voraussetzungen von DLTs profitieren kann. Die Erkenntnisse sind, dass sich DLT als Technologie für dezentrale und asynchrone IOT-Anwendungsfälle eignet, an denen mehrere, sich gegeneinander nicht vertrauende Parteien teilnehmen. Die Basis aller nicht-funktionalen Anforderungen sind für alle \ac{DLT}-\ac{IOT}-Anwendungsfälle gleich.
\end{otherlanguage}
