\chapter{Diskussion}
\label{ch:discussion}
Dieses Kapitel befasst sich mit der zu eingangs aufgestellten These:
\begin{quote}
  '\ac{DLT} eignet sich als Technologie für \ac{IOT} und die nicht-funktionalen Anforderungen sind für alle \ac{DLT}-\ac{IOT}-Anwendungsfälle gleich.'
\end{quote}


\section{Wiederaufnahme These Teil 1: Eignung für IOT?}
\label{sec:discussion:part1}
Der erste Teil der These stellt die Erwartung auf, dass \ac{DLT} eine geeignete Technologie darstellt, um \ac{IOT}-Anwendungsfälle umzusetzen. Nach ausführlicher Bearbeitung und Überprüfung in dieser Arbeit kann zunächst festgehalten werden, dass \ac{DLT} eine geeignete Technologie für die Umsetzung dieses Anwendungsfalles darstellt. Die Umsetzung ist erfolgreich; eine geeignete Implementierung wurde vorgenommen, die den Anwendungsfall auf Basis von Ethereum prototypisch evaluiert. Eine generelle Aussage für das gesamte Anwendungsgebiet \ac{IOT} ist damit nicht getroffen und kann auch nicht vollständig validiert werden. Anwendungsfälle, die ähnliche Charakteristika aufzeigen wie die vorliegende Umsetzung, können mit hoher Sicherheit von einer \ac{DLT}-Implementierung profitieren; die Vorteile hierfür wurden in Kapitel \ref{subsec:fundamentals:dlt:protocol} aufgezeigt. Dem entgegen stehen zeitkritische oder Performanz-lastige \ac{IOT}-Anwendungen, die nicht auf Basis eines \acp{DLT} implementiert werden können, da beispielsweise die benötigte Echtzeitkommunikation nicht gegeben ist. Darüber hinaus sind auch Anwendungen, die nur einen einzigen Stakeholder betreffen, oder nur an einer Lokalität ausgeführt werden, nicht für \acp{DLT} ausgelegt, da dabei nicht von der Dezentralität und dem No-Trust Environment profitiert werden kann. Demnach wird der erste Teil der These korrigiert, um die Erkenntnisse dieser Arbeit korrekt widerzuspiegeln. Damit ergibt sich folgende, mit dieser Arbeit validierte Aussage:
\begin{quote}
  '\ac{DLT} eignet sich als Technologie für dezentrale und asynchrone \ac{IOT}-Anwendungsfälle, an denen mehrere, sich gegeneinander nicht vertrauende Parteien teilnehmen.'
\end{quote}

\section{Wiederaufnahme These Teil 2: Technische Anforderungen immer gleich?}
\label{sec:discussion:part2}
Der zweite Teil der These stellt die Erwartung auf, dass die nicht-funktionalen Anforderungen für alle \ac{DLT}-\ac{IOT}-Anwendungsfälle gleich sind. Wie bereits erläutert, beschränkt sich die Eignung von \ac{DLT} auf eine Untermenge aller \ac{IOT}-Anwendungsfälle. Diese weisen die gleichen oder zumindest ähnliche Charakteristika auf wie der vorliegende Anwendungsfall. Die in Tabelle \ref{tab:dlt_relevant} als relevant klassifizierten, nicht-funktionalen Anforderungen betreffen die Unterbereiche Sicherheit und Zuverlässigkeit. Namentlich handelt es sich um die Eindeutigkeit, Manipulationssicherheit, Zugriffsbeschränkung und Performanz; dies sind Eigenschaften, die jede produktiv eingesetzte \ac{IOT}-Anwendung garantieren muss. Die Performanz kann dabei abhängig vom Anwendungsfall variieren, ist allerdings durch die Wahl des \acp{DLT} variabel gestaltbar.\\
Es handelt sich um Anforderungen, die einerseits von jedem \ac{IOT}-Anwendungsfall gefordert werden, aber auf der anderen Seite keine Vollständigkeit abbilden. Es ist denkbar, dass es Anwendungsfälle gibt, die über die genannten, nicht-funktionalen Anforderungen hinaus weitere Besonderheiten mit sich bringen. Demnach sind Eindeutigkeit, Manipulationssicherheit, Zugriffsbeschränkung und Performanz als Basis eines jeden \ac{DLT}-\ac{IOT}-Anwendungsfalles zu verstehen und damit für alle gleich. Auf Grundlage dieser Ausführung wird der zweite Teil der These wie folgt geändert, um den Erkenntnissen gerecht zu werden:
\begin{quote}
  '... die Basis aller nicht-funktionalen Anforderungen sind für alle \ac{DLT}-\ac{IOT}-Anwendungsfälle gleich.'
\end{quote}


\section{Ergebnis}
\label{sec:discussion:result}
Die Überprüfungen beider Teile der Ausgangsthese führt zu folgendem Gesamtergebnis, welches lautet:
\begin{quote}
  '\ac{DLT} eignet sich als Technologie für dezentrale und asynchrone \ac{IOT}-Anwendungsfälle, an denen mehrere, sich gegeneinander nicht vertrauende Parteien teilnehmen und die Basis aller nicht-funktionalen Anforderungen sind für alle \ac{DLT}-\ac{IOT}-Anwendungsfälle gleich.'
\end{quote}
