\chapter{Ergebnisse \& Fazit}
\label{ch:results}
In dieser Arbeit wurde ein beispielhafter \ac{IOT}-Anwendungsfall erarbeitet und in nachvollziehbaren Schritten auf \ac{DLT}-Tauglichkeit evaluiert. Dazu wurden Anforderungen aufgestellt, klassifiziert und anhand verschiedener \ac{DLT}-Lösungen theoretisch überprüft. Die prototypische Verprobung wurde daraufhin auf Basis von Ethereum durchgeführt. Mit dieser Implementierung wurden die Anforderungen weitestgehend erfüllt, Abweichungen davon wurden im vorherigen Kapitel genannt und begründet. Die folgenden Ausführungen untergliedern die Ergebnisse in sieben Kategorien und fassen die Erkenntnisse zusammen.


\section{Anforderungsklassifizierung}
\label{sec:results:classification}
Das Klassifizierungsmodell aus Kapitel \ref{sec:requirements:model} wurde ursprünglich erarbeitet, um die Anforderungen des Anwendungsfalles in Klassen einzuteilen, damit auf dieser Ebene eine Aussage zur \ac{DLT}-Relevanz getroffen werden kann. Das erarbeitete Modell eignete sich gut, um Anforderungen einfach und aus verschiedenen Perspektiven heraus zu identifizieren. Die These, damit auch eine vereinfachte Klassifizierung vornehmen zu können, wurde verworfen, da das Modell die Einordnung nicht erleichterte und für diesen Aspekt nicht den erhofften Mehrwehrt erbrachte. Die Klassifizierung als solche konnte allerdings genutzt werden, um möglichst alle Bereiche und Perspektiven verschiedener Stakeholder in Bezug auf die Anforderungsermittlung abzudecken. Die Überprüfung auf \ac{DLT}-Relevanz muss nach wie vor auf der Anforderungsebene durchgeführt und für jede Anforderung separat überprüft werden.

\section{Machbarkeit}
\label{sec:results:feasibility}
Dieser Anwendungsfall ist für die Umsetzung auf Basis eines \acp{DLT} geeignet. Die Implementierung wurde erfolgreich umgesetzt, getestet und damit die Machbarkeit nachgewiesen. Inwieweit die generelle Machbarkeit über diesen Anwendungsfall hinaus geht und ob es möglich und sinnvoll ist, jegliche Art von \ac{IOT}-Anwendungsfällen auf Basis von \ac{DLT} umzusetzen, ist damit nicht geklärt. Die Diskussion über diese Thematik wird in Kapitel \ref{sec:discussion:part1} wiederaufgenommen.

\section{Kosten}
\label{sec:results:costs}
In Kapitel \ref{sec:results:costs} werden die Kosten des Anwendungsfall detailliert aufgeschlüsselt. Die Endausbaustufe sieht 400.000 Benutzer und 10.000 Kaffeemaschinen vor, die durch den Anwendungsfall prozessiert werden. Damit ergibt sich ein errechneter Gesamtumsatz von 82.000.000 € pro Jahr bei jährlichen Gebühren von 78.000 €. Um alle Verträge mit den Benutzern initial abzuschließen ergeben sich Kosten in Höhe von 4.200 € für den Hersteller. Aus Benutzersicht belaufen sich die monatlichen Gebühren zur Nutzung auf 3 Cent und initiale Kosten für den Vertragsabschluss von 35 Cent. Mit steigenden Nutzer-, Vertrags- und Gerätezahlen auf der Plattform werden nach aktueller Implementierung diese Preise steigen\footnote{Jede Smart-Contract Rechenoperation benötigt GAS. Eine Schleife, die alle existierenden Verträge durchläuft und überprüft, hat mit steigender Vertragszahl auch steigende Schleifendurchläufe.}; eine effizientere Implementierung ist denkbar und könnte dies vermeiden.

\section{Performanz}
\label{sec:results:performance}
Die Performanz aus Benutzersicht muss in zwei Nutzungsphasen unterteilt werden: Das initiale Vertragserstellen ist ein asynchroner Prozess, der das Zutun beider Vertragsparteien erfordert und demnach nicht exakt gemessen werden kann. Überträgt man diesen Prozess auf den analogen Prozess einer Angebotsanfrage bis hin zum Vertragsabschluss, so stellt man hier allein durch die Digitalisierung einen deutlichen Geschwindigkeitszuwachs fest. Die reine Prozessierungsdauer von der Anfrage bis zum Zustandekommen des Vertrages entspricht drei onchain Transaktionen und damit im Durchschnitt der Größenordnung von wenigen Minuten. Die eigentliche Nutzung, also das Interagieren des Benutzers mit der Kaffeemaschine, erfolgt instantan und bietet dem Benutzer eine optimale Performanz und damit ein gutes Benutzererlebnis.\\
Aus Herstellersicht existieren keine zeitkritischen Prozesse, da es nicht darauf ankommt, dass ein Vertrag innerhalb von Sekunden zustandekommt, da der Zeitpunkt der Vertragsbestätigung durch den Kunden nicht beeinflussbar ist. Das Einlösen einer Quittung ist mit einer onchain Transaktion binnen weniger Sekunden (weniger als einer Minute) erfolgt und befindet sich damit in einem akzeptablen zeitlichen Rahmen.

\section{Sicherheit}
\label{sec:results:Security}
Dem Anwendungsfall liegt ein Rollen- und Benutzermanagement zugrunde, das Berechtigungen überprüft und Zugriffe beschränkt. Der zentrale Sicherheitsaspekt ist der Private-Key jedes Benutzers. Es existiert in der aktuellen Implementierung kein Mechanismus, welcher den Verlust oder den Diebstahl eines Private-Keys kompensiert. Die Ethereum Blockchain wird als extrem sicher eingestuft; Manipulationen oder Angriffe auf das System sind nur mit extrem hohem Aufwand und immensen Kosten verbunden. Die Sicherheit der implementierten Smart-Contracts muss von Experten vor einem Produktiveinsatz eingehend überprüft werden, damit Schäden durch Dritte oder Fehler im Code vermieden werden können.

\section{Vorteile zu klassischen Ansätzen}
\label{sec:results:advantages}
Es wurde gezeigt, dass die Implementierung des \ac{IOT}-Anwendungsfalles durch den Einsatz der \ac{DLT}-Technologie in verschiedenen Bereichen profitieren konnte. Ein schnelles Prototyping konnte durch die bereitgestellte Infrastruktur der Blockchain sowie den umfangreichen Entwicklungswerkzeugen schnell umgesetzt werden, sodass frühzeitig erste, vorzeigbare Ergebnisse vorlagen. Durch das Bereitstellen eines vollumfänglichen Kommunikationsprotokolls (vgl. Kap. \ref{subsec:fundamentals:dlt:protocol}) konnte sich bei der Entwicklung auf die Kernpunkte konzentriert werden. Die enorm hohe Ausfall- und Manipulationssicherheit sind zwei große Pluspunkte, mit denen die Implementierung durch den Einsatz der Ethereum-Blockchain implizit aufwarten kann.

\section{Datenschutz \& Privatsphäre}
\label{sec:results:privacy}
Die Aspekte Datenschutz und Privatsphäre wurden in der vorliegenden Implementierung nicht berücksichtigt. Das Konzept sieht allerdings Mechanismen vor, die für die notwendige Sicherheit sorgen und die Mängel (vgl. Kap. \ref{subsub:implementation:requirements:encryption:privacy}) adressieren; darüber hinaus werden in Kapitel \ref{ch:perspective} weitere Ansätze aufgezeigt, um den Datenschutz und die Privatsphäre der Stakeholder zu wahren.

\section{Implementierungsfortschritt}
\label{sec:results:progress}
Ziel dieser Arbeit war es, zu zeigen, dass der der ausgewählte \ac{IOT}-Anwendungsfall durch den Einsatz eines \acp{DLT} umsetzbar ist. Dadurch konnte der ursprünglich skizzierte Anwendungsfall (vgl. Kap. \ref{ch:iot_usecase}) nicht vollumfänglich umgesetzt werden. Die Dazunahme von Service-Providern und Lieferanten oder darüber hinaus noch weiteren Stakeholdern ist eine Möglichkeit, den Anwendungsfall weiter auszubauen. Dazu würden weitere Vertragstypen wie Service- und Lieferverträge mittels eigener Smart-Contracts implementiert und an das bestehende System angehängt werden.
