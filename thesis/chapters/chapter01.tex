\chapter{Einleitung}
\label{ch:intro}
Das digitale Zahlungsmittel Bitcoin, ein für \ac{IOT}-Anwen\-dungen geschaffenes System wie IOTA oder ein komplexes System wie Ethereum, auf welchem sich umfangreiche Geschäftslogiken mittels Smart-Contracts umsetzen lassen: Die noch recht Junge Technologie hinter Block\-chain-Lösungen - oder allgemeiner \ac{DLT}-Lösungen - greift auf grundlegende Ansätze zurück, die keineswegs neu sind:\\
Vor über 1500 Jahren suchten die Einwohner der Insel Yap im Westpazifik östlich der Philippinen eine Lösung für ihr Zahlungsmittel-Problem. Ihre Währung - Rai - waren großen runde Steintafeln, die teils mehrere Tonnen wiegen konnten und als alltägliches Zahlungsmittel ungeeignet waren. Die Dorfältesten merkten sich, wem welcher Rei-Stein gehörte. Fand eine Transaktion statt, so wurde diese den Dorfältesten mitgeteilt. Das Wissen, wem welcher Stein gehörte und wer von wem etwas gekauft hatte, war über mehrere Köpfe verteilt; die dezentrale Informationsspeicherung war geboren \cite{forbes2018}. Schaut man in die jüngere Vergangenheit, so lässt sich die historische Entwicklung fortführen: Hash-Bäumeauch Merkle-Trees genannt, wurden 1979 von Ralph Merkle erfunden und sind heute eine der grundlegenden Bauteile einer Blockchain. 1991 beschrieben die Forscher Stuart Haber und W. Scott Stornetta die Idee hinter der Blockchain-Technologie \cite{binance2019}. 2008 veröffentlichte Satoshi Nakamoto das Whitepaper zu Bitcoin, dicht gefolgt von Ethereum, welches 2013 von Vitalik Buterin erstmals vorgestellt wurde.\\
Und wie sieht es heute aus? Mittlerweile ist ein ganzes Ökosystem an Block\-chain- und \ac{DLT}-Lösungen entstanden; die Website coinmarketcap.com listet beinahe 5000 Kryptowährungen mit einer Marktkapitalisierung von mehr als 200 Milliarden US-Dollar (Stand: 12/2019). Mittels sogenannter Smart-Contracts lassen sich komplexe Geschäftslogiken auf der Blockchain umsetzen und kombinieren, sodass ganze Anwendungen auf Blockchain-Basis erstellt werden können. Solche \ac{DApp}s sind ein Schritt in die Zukunft hin zum Web 3.0 - auch Semantic Web genannt. Dabei handelt es sich um die nächste Evolutionsstufe des World-Wide-Webs, in dem Maschinen und Programme in der Lage sind, Anfragen und Interaktionen mit Menschen eine Semantik zuzuordnen. Das ist dann der Moment, indem die Kaffeemaschine den Kaffeekonsum des Benutzers analysiert und morgens den Kaffee selbstständig serviert. Produkte werden automatisch nachbestellt, da die Maschine weiß, wann der Kaffee aufgebraucht sein wird. Und wenn Besuch kommt, dann kann die Maschine den kalkulierten Verbrauch automatisch anpassen. Diese Zukunft wird möglich gemacht durch die Umsetzung des Web3.0 - unterstützt durch die Blockchain-Technologie und dezentrale Anwendungen.

%
% Section: Motivation
%
\section{Motivation \& Problemstellung}
\label{sec:intro:motivation}
Das Telekommunikationsunternehmen Cisco prognostiziert, dass bis zum Jahr 2030 mehr als 500 Milliarden mit dem Internet verbundene \ac{IOT}-Geräte in verschiedenen Bereichen unseres alltäglichen Lebens Einzug erhalten haben werden \cite{cisco2016}. Vernetzte Dinge unseres Alltags wie Kühlschranke, Kaffeemaschinen, aber auch die aus dem Business-Umfeld automatisierte Supply-Chain oder eine Smart-City sind nur einige wenige Beispiele dieses Geschäftsfeldes. Das Konzept von \ac{IOT} ist nach wie vor sehr theoretisch, obwohl bereits einige Anwendungsfälle erarbeitet wurden. Um das große Potential von \ac{IOT} vollumfänglich nutzbar zu machen und entsprechende Visionen umzusetzen, muss eine passende Backbone-Lösung für solche Anwendungsfälle bereitgestellt werden. Viele verschiedene Hersteller und Service-Provider benötigen eine einheitliche Plattform, auf der sie ihre \ac{IOT}-Geräte, Services, Geschäftslogiken und Kunden miteinander vernetzen können sowie die Integration eines sicheren Bezahlsystems. Es stellt sich die Frage, ob und inwiefern die zwei innovativen Technologien \ac{DLT} und \ac{IOT} voneinander profitieren können und ob sich \ac{DLT} als skalierende, performante und sichere Backbone-Technologie für \ac{IOT}-Anwendungsfälle eignet.

%
% Section: Zielsetzung \& Zielgruppe
%
\section{Zielsetzung \& Zielgruppe}
\label{sec:intro:goal}
Das Ziel dieser Arbeit ist die Untersuchung und prototypische Verprobung der folgenden These: '\ac{DLT} eignet sich als Backbone-Technologie für \ac{IOT} und die technischen / nicht-funktionalen Anforderungen sind für alle Anwendungsfälle gleich'. Es wird gezeigt, inwieweit sich die Technologie \ac{DLT} als Backbone-System für \ac{IOT}-Anwendungsfälle eignet, welche Anforderungen dafür erfüllt sein müssen, und welche Implementierung für die Umsetzung in Frage kommt. Dazu im Verlauf der Arbeit das zu lösende Problem genauer spezifiziert und herausgearbeitet: Nachdem ein konkreter \ac{IOT}-Anwendungsfall vorgestellt wurde und alle \ac{DLT}-relevanten Anforderungen ermittelt und evaluiert sind, ergibt sich das konkrete Problem als die Umsetzung eines speziellen \ac{IOT}-Anwendungsfalls mit einer konkreten \ac{DLT}-Lösung. Der explizite Lösungsraum, also welche Anforderungen umgesetzt werden sollen, wird zuvor genau beschrieben. Das Problem gilt als gelöst, sobald die zuvor abgeleiteten Anforderungen mit dem \ac{PoC} erfüllt werden können. Abschließend wird die eingangs formulierte These diskutiert und evaluiert.\\
Diese Arbeit richtet sich an IT-Spezialisten aus dem Umfeld \ac{DLT} und \ac{IOT}, die sich über die Synergie beider Konzepte informieren, sowie Fachleuten aus der Industrie, die entsprechende \ac{IOT}-Anwendungsfälle ausarbeiten möchten. Ein solides Grundverständnis für die grundlegenden Konzepte und Wordings wird an dieser Stelle vorausgesetzt; auf entsprechende Grundlagenliteratur wird gegebenenfalls verwiesen. In Kapitel \ref{ch:fundamentals} wird das benötigte Basiswissen vermittelt, welches für das Verständnis dieser Arbeit und der Zusammenhäng von Nöten ist.

%
% Section: Methodik
%
\section{Methodik \& Vorgehen}
\label{sec:intro:methodology}
In dieser Arbeit werden die Themenbereiche '\ac{DLT}' und '\ac{IOT}' vorgestellt, klassifiziert und in das \ac{OSI} Referenzmodell eingeordnet. Die Synergie beider Bereiche wird herausgearbeitet und es wird dem Leser vorgestellt, wie diese Technologien voneinander profitieren können. Ein beispielhafter \ac{IOT}-Anwendungsfall wird entwickelt und eine detaillierte Auflistung aller Anforderungen erarbeitet. Im nächsten Schritt werden die die ermittelten Anforderungen schrittweise auf eine Untermenge von fundamentalen Anforderungen reduziert, die relevant für \ac{IOT} in Verbindung mit \ac{DLT} sind. Mehrere, am Markt etablierte \ac{DLT} Anwendungen werden anschließend vorgestellt und auf Basis dieser Anforderungsuntermenge evaluiert. Es wird geprüft, ob und inwieweit sie sich als Backbone-Lösung für den \ac{IOT}-Anwendungsfall qualifizieren. Die vielversprechendste Lösung wird in einem \ac{PoC} prototypisch umgesetzt, um die Anforderungsliste zu evaluieren. Es wird gezeigt, dass die gewählte \ac{DLT} zielbringend als \ac{IOT} Backbone-Lösung eingesetzt werden kann. Abschließend wird diskutiert, dass die nicht-funktionalen Anforderungen für \ac{DLT}-geeignete \ac{IOT}-Anwendungsfälle, unabhängig vom tatsächlichen Anwendungsfall selbst, stets die gleichen sind.

Diese Arbeit zeigt die Eignung von verschiedenen \ac{DLT}s als Backbone-Lösung für \ac{IOT}-Anwendungsfälle anhand eines beispielhaften \ac{PoC}s. Es werden nur solche Bereiche von \ac{IOT} betrachtet, die auch grundsätzlich für die Implementierung auf \ac{DLT}s geeignet sind. Es gibt darüber hinaus weitere Bereiche, die sich nicht eignen, um auf \ac{DLT}s umgesetzt zu werden und müssen auf einer anderen technologischen Basis implementiert werden. Des weiteren wird die in dieser Arbeit durchgeführte Analyse anhand eines \ac{PoC} belegt. Aufgrund von Restriktionen wie der Zeitlimitierung und der praktischen Umsetzbarkeit könnten unter Umständen nicht alle fundamentalen Anforderungen gezeigt werden, die für einen \ac{IOT}-\ac{DLT}-Anwendungsfall erfüllt sein müssen.


%
% Section: Abgrenzung dieser Arbeit
%
\section{Abgrenzung dieser Arbeit}
\label{sec:relatedwork:differentiation}
\textbf{\textcolor{red}{todo}}


%
% Section: Aufbau dieser Arbeit
%
\section{Aufbau dieser Arbeit}
\label{sec:intro:structure}
\textbf{\textcolor{red}{todo}}
