\chapter{Ausblick}
\label{ch:perspective}
Mit dieser Arbeit wurde die Machbarkeit von \ac{IOT}-Anwendungsfällen durch \ac{DLT} nachgewiesen. Die vorliegende Implementierung kann als Grundlage für weitere Ausbaustufen des Anwendungsfalls dienen und eine Vorlage für die Umsetzung anderer \c{IOT}-Anwendungsfälle auf Basis eines \acp{DLT} sein. Mögliche Verbesserungen, Weiterentwicklungen und alternative Ansätze werden in diesem Ausblick aufgezeigt und vorgestellt. Dabei wurden Handlungsfelder identifiziert, die Optimierungspotential bieten und in denen weitere Forschung betrieben werden kann:\\


\section{Kostensenkung}
\label{sec:perspective:costs}
Um die Akzeptanz und die Verbreitung von \ac{IOT}-Anwendungsfällen weiterhin zu vergrößern, ist es wichtig, die Kosten für die Nutzung zu reduzieren. Dies betrifft sowohl die Kosten aus Benutzersicht als auch aus Sicht des Herstellers. Mit steigender Anzahl an \ac{IOT}-Geräten wird es für den dauerhaften Fortbestand eines Anwendungsfalles existenziell, die Kosten so gering wie möglich zu halten. Die Umsetzung einer dezentralen Public-Blockchain ohne Transaktionskosten ist derzeit noch nicht erreicht, um solche Anwendungsfälle zu begünstigen. Ein möglicher Forschungsaspekt liegt in der Erarbeitung eines Konsensmodells, dass ohne Transaktionskosten auskommt. Solange dies nicht erreicht ist, kann man zumindest versuchen, die Zahl der Transaktionen und damit auch die Nutzungskosten so gering wie möglich zu halten. Jede Transaktion und jede Codezeile innerhalb eines Smart-Contracts kostet Geld. Um die Vorteile von onchain Transaktionen nicht zu verlieren und dennoch die genannten Punkte sinnvoll zu adressieren, bietet sich ein hybrides Modell aus on- und offchain Verarbeitung an. Ein möglicher Ansatz wäre die dezentrale Datenhaltung mittels \ac{IPFS}: Durch ein ausgeklügeltes Versionierungssystem in Kombination mit kryptographischen Hashfunktionen können eindeutige Hashes über Dateien und Informationen erzeugt und mittels dieser Hashes eine Verknüpfung zwischen onchain Informationsverarbeitung und offchain Datenhaltung geschaffen werden.\\
Eine andere, temporäre Lösung könnte mit der Verlagerung der Kosten einhergehen: Das Implementieren von Proxy Smart-Contracts, die als Relay fungieren und es erlauben, signierte Transaktionen im Namen eines Dritten an das Blockchain-Netzwerk zu senden. Solche Metatransaktionen ermöglichen, dass signierte Transaktionen von Alice über den Account von Bob an den Distributed Ledger transferiert werden können. Im Proxy Smart-Contract ist festgelegt, welcher Account in wessen Namen Transaktionen tätigen darf; diese Berechtigungen können jederzeit angepasst werden. Dabei muss der ursprüngliche Absender die Transaktion nach wie vor signieren, lediglich die Übermittlung und damit das Tragen der Transaktionskosten läuft über einen Dritten. Im konkreten Anwendungsfall würde dies das Onboarding des Benutzers stark erleichtern, da dieser zur Nutzung der Plattform nicht zunächst Ether erwerben müsste. Darüber hinaus kann dadurch die Akzeptanz der Benutzer erhöht werden. Die Transaktionsgebühren können durch den Hersteller oder Betreiber der Plattform getragen und somit die Benutzbarkeit drastisch erhöht werden.

\section{Privatsphäre \& Datenschutz}
\label{sec:perspective:privacy}
Durch die onchain Datenhaltung und Informationsverarbeitung mittels Smart-Contract können alle Vorteile, die ein \ac{DLT} bietet (vgl. Kap. \ref{subsec:fundamentals:dlt:protocol}), zielbringend eingesetzt werden. Dies bringt allerdings auch Nachteile mit sich, die sich vor allem negativ auf die Privatsphäre und den Datenschutz auswirken - alle Informationen sind öffentlich und für jedermann einsehbar. Dies ist besonders bei personenbezogenen Daten nicht wünschenswert und muss in Produktivumgebungen verhindert werden. Durch die dezentrale Datenhaltung mittels \ac{IPFS} oder ähnlichen Technologien wurde ein erster Ansatz aufgezeigt, Daten auch außerhalb der Blockchain aufzubewahren. Dennoch hat der Benutzer keine volle Kontrolle mehr über seine Daten, da diese den eigenen Zugriffsbereich verlassen und über das Internet repliziert werden. Das Konzept von Zero-Knowledge Proofs geht hierbei deutlich weiter: Eigene Daten werden nicht mehr an Dritte übertragen, sondern es wird ein mathematisch verifizierbarer Beweis erstellt, ohne den genauen Inhalt des Beweises preiszugeben \cite{zeroknowledge2020}. So können beispielsweise Informationen zur eigenen Identität oder Zugangsberechtigungen mittels Zero-Knowledge Proof nachgewiesen werden. Konkret bedeutet das, dass das Rollenmanagement des vorliegenden Anwendungsfalles durch das Oracle entfallen und man in einer Rolle agieren könnte, ohne diese oder die eigene Identität preiszugeben. Auch könnten die Quittungen des Payments und damit die Nachvollziehbarkeit des Kaffeekonsums verschleiert werden, indem diese durch Zero-Knowledge Proofs ersetzt würden und damit lediglich der Beweis erbracht würde, dass ein Account berechtigt ist, eine bestimmte Menge Ether zu empfangen.\\
Ein weiterer, wichtiger Aspekt in Bezug auf Datenschutz ist das Konzept des Private-Keys: Das Sicherheitskonzept basiert vollständig auf der Annahme, dass der Private-Key jedes Stakeholders geheim und unzugänglich für Dritte ist. Dieser \ac{SPoF} kann durch besondere Maßnahmen vermieden werden. Ein möglicher Ansatz wäre der Einsatz von onchain-Wallets, die mittels Smart-Contracts implementiert werden und Sicherheitskonzepte wie Guards zulassen: Ein Guard ist ein anderer Account, dem der Inhaber des Wallets vertraut und der im Falle des Verlusts des Private-Keys oder eines Angriffes das Konto sperren, einen neuen Private-Key für den Benutzer hinterlegen oder andere Sicherheitsmaßnahmen durchführen kann. Dabei ist der Einsatz von mehreren Guards möglich, wobei die Berechtigungen dieser angepasst und durch besondere Bedingungen beschränkt werden können. Dadurch könnten präventiv Maßnahmen gegen den Missbrauch von Datenschutz-relevanten Informationen durchgeführt werden, falls der Private-Key eines Stakeholders korrumpiert wird.

% \item[Quittung liegt nur lokal auf Gerät!]
% \item[Identitäten] Das Identitätsmanagement wird in der vorliegenden Implementierung durch einen Oracle Smart-Contract durchgeführt. Dieser hält Informationen zu Identitäten, deren Rollen und Berechtigungen. Dadurch sind sämtliche Informationen öffentlich zugänglich und für jedermann einsehbar. Die Einführung von dezentralen Identitäten in Kombination mit \ac{VC}s kann dieser Problematik entgegenwirken und bringt darüber hinaus weitere Vorteile mit sich.
% Konkret bedeutet das für den Anwendungsfall, dass beispielsweise mit \ac{DID}s der Private-Key des Herstellers nicht mehr auf Maschine vorliegen muss, damit diese im Namen des Herstellers die Quittung einlösen kann. Zukünftig kann der Hersteller der Maschine ein \ac{VC} ausstellen, welches der Maschine erlaubt, die Quittung im Namen des Herstellers einzulösen. Darüber hinaus lässt es komplexere Szenarien zu: Ein Mietvertrag kann von einem Unternehmen erstellt und von dessen Mitarbeitern genutzt werden, indem jeder Mitarbeiter ein \ac{VC} des Unternehmens erhält, dort zu arbeiten und berechtigt ist, die Kaffeemaschine zu benutzen. Die Bezahlung kann dann entweder jeder Mitarbeiter selbst übernehmen oder erneut durch ein \ac{VC} des Unternehmens über dieses abrechnen.
