%*******************************************************
% Abstract in English
%*******************************************************
\pdfbookmark[1]{Abstract}{Abstract}


\begin{otherlanguage}{american}
	\chapter*{Abstract}
	The internet of things (IOT) as an emerging technology enables a fully-automated machine-to-machine (M2M) communication to make our everydays lifes easier: smart homes, the vision of smart cities or latency sensible connected cars are only some examples of possible research areas in this business. A huge amount of data needs to be processed, transmitted and stored. The presentation of the next generation mobile standard 5G and high-performance devices enable the transmitting and processing. To securly store the data a suitable solution must be found.\\
	Besides that the distributed ledger technology (DLT) earns high attention due to the newly developed use cases which benefit from the trustless and decentralized properties of DLT.\\
	It is obvious that an analyzation of those two relatively new technologies might result in a benefit for IOT and DLT. There could be a synergy so that both technologies could complement each other.\\
	This work evaluates a selection of well-established DLTs for their suitability in the context of a representative IOT use case. Therefore the requirements for the use case are analyzed and applied to the DLTs. In a next step the DLTs are evaluated and ranked by the fulfillment of the requirements. A prototypical implementation of the IOT use case with the best suitable DLT is created. For a real-world adaption the use case is created by different IOT devices and sensors that transmit data over the DLT to other devices. Load-tests are used to check the actual performance.\\
	As a result this work presents a comprehensible list of DLTs with an evaluation and ranking for their IOT-suitability in the M2M-area. On top that the feasibility of IOT use cases with DLT technology is shown.
\end{otherlanguage}
