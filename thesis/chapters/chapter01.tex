\chapter{Einleitung}
\label{ch:intro}
Lorem ipsum at nusquam appellantur his, labitur bonorum pri no \citep{dueck:trio}. His no decore nemore graecis. In eos meis nominavi, liber soluta vim cu. Sea commune suavitate interpretaris eu, vix eu libris efficiantur.

%
% Section: Motivation
%
\section{Motivation \& Problemstellung}
\label{sec:intro:motivation}
\graffito{Note: The content of this chapter is just some dummy text. It is not a real language.}
\lipsum[1-1]
Der Begriff Blockchain - oder allgemeiner \ac{DLT} - wird heute meist nur synonym für Bitcoin oder dessen Artgenossen verwendet. Dabei handelt es sich um eine Technologie, die bereits mehr und mehr Einzug in unseren Alltag erhält: Sei es ein digitales Zahlungsmittel wie Bitcoin, ein global verteilter Supercomputer wie Golem oder ein komplexes System wie Ethereum, auf welchem sich umfangreiche Geschäftslogiken mittels Smart-Contracts umsetzen lassen; die Möglichkeiten und Anwendungsfälle scheinen schier unbegrenzt zu sein. Es entstehen fast täglich neue Anwendungsfälle; unter Anderem auch im Umfeld der noch recht jungen Technik hinter \ac{IOT}.\\
Das große Telekommunikationsunternehmen Cisco progostiziert, dass bis zum Jahr 2030 mehr als 500 Milliarden mit dem Internet verbundene IOT-Geräte in verschiedenen Bereichen unseres alltäglichen Lebens Einzug erhalten haben werden\footnote{https://www.cisco.com/c/dam/en/us/products/collateral/se/internet-of-things/at-a-glance-c45-731471.pdf}. Vernetzte Dinge unseres Alltags wie Kühlschranke oder Kaffeemaschinen, aber auch die aus dem Business-Umfeld autmatisierte Supply-Chain oder eine Smart-City sind nur einige wenige Beispiele dieses Geschäftsfeldes. Das Konzept von IOT ist nach wie vor sehr theoretisch, obwohl bereits einige Anwendungsfälle erarbeitet wurden. Um das große Potential von IOT vollumfänglich nutzbar zu machen, muss eine passende Backbone-Lösung für solche Anwendungsfälle bereitgestellt werden. Viele verschiedene Hersteller und Service-Provider benötigen eine einheitliche Plattform, auf der sie ihre IOT-Geräte, Services, Geschäftslogiken und vor allem ihre Kunden miteinander vernetzen sowie ein optimales Bezahlsystem zur Verfügung stellen können. Es stellt sich die Frage, ob und inwiefern diese zwei innovativen Technologien voneinader profitieren können und ob \ac{DLT} als hoch-skalierende, performante und sichere Backbone-Technologie eignet.

%
% Section: Zielsetzung \& Zielgruppe
%
\section{Zielsetzung \& Zielgruppe}
\label{sec:intro:goal}
In dieser Arbeit wird die folgende These untersucht: 'DLT eignet sich als Backbone-Technologie für IOT und die technischen / nicht-funktionalen Anforderungen sind für alle Anwendungsfälle gleich'. Demnach wird untersucht, inwieweit sich die Technologie \ac{DLT} als Backbone-System für \ac{IOT}-Anwendungsfälle eignet, welche Anforderungen dafür erfüllt sein müssen, und  Diese Arbeit richtet sich an IT-Spezialisten aus dem Umfeld \ac{DLT} und \ac{IOT}, die sich über die Synergie beider Konzepte informieren möchten, sowie Fachleuten aus der Industrie, die entsprechende \ac{IOT}-Anwendungsfälle ausarbeiten möchten. Ein solides Grundverständnis für die grundlegenden Konzepte und Wordings wird an dieser Stelle vorausgesetzt.

%
% Section: Methodik
%
\section{Methodik \& Vorgehen}
\label{sec:intro:methodology}
\lipsum[1-1]

In dieser Arbeit werden die Themenbereiche 'DLT' und 'IOT' vorgestellt, klassifiziert und in das OSI-Referenzmodell eingeordnet. Die Synergie beider Bereiche wird herausgearbeitet und es wird dem Leser vorgestellt, wie diese Technologien voneinander profitieren können. Ein beispielhafter IOT-Anwendungsfall wird entwickelt und eine detaillierte Auflistung aller Anforderungen erarbeitet. Im nächsten Schritt werden die die ermittelten Anforderungen schrittweise auf eine Untermenge von fundamentalen Anforderungen reduziert, die relevant für IOT in Verbindung mit DLT sind. Mehrere, am Markt etablierte DLT Anwendungen werden anschließend vorgestellt und auf Basis dieser Untermenge evaluiert. Es wird geprüft, ob und inwieweit sie sich als Backbone-Lösung für den IOT-Anwendungsfall qualifizieren. Die vielversprechendste Lösung wird in einem PoC umgesetzt, um die Anforderungsliste zu evaluieren. Es wird gezeigt, dass die gewählte DLT zielbringend als IOT Backbone-Lösung eingesetzt werden kann. Abschließend wird gezeigt, dass die nicht-funktionalen Anforderungen für DLT-geeignete IOT-Anwendungsfälle, unabhängig vom tatsächlichen Anwendungsfall selbst, stets die gleichen sind.

Diese Arbeit zeigt die Eignung von verschiedenen DLTs als Backbone-Lösung für IOT-Anwendungsfälle anhand eines beispielhaften PoCs. Es werden nur solche Bereiche von IOT betrachtet, die auch grundsätzlich für die Implementierung auf DLTs geeignet sind. Es gibt darüber hinaus weitere Bereiche, die sich nicht eignen, um auf DLTs umgesetzt zu werden und müssen auf einer anderen technologischen Basis implementiert werden. Des weiteren wird die in dieser Arbeit durchgeführte Analyse anhand eines PoC belegt. Aufgrund von Restriktionen wie der Zeitlimitierung und der praktischen Umsetzbarkeit könnten unter Umständen nicht alle fundamentalen Anforderungen gezeigt werden, die für einen IOT-DLT-Anwendungsfall erfüllt sein müssen.


%
% Section: Aufbau dieser Arbeit
%
\section{Aufbau dieser Arbeit}
\label{sec:intro:structure}
\lipsum[1-1]
