\chapter{Verwandte Forschungsarbeiten}
\label{ch:relatedwork}
Die Themenbereiche \ac{DLT} und \ac{IOT} sind bereits an vielen Stellen untersucht und beschrieben worden. Darüber hinaus existieren auch einige Werke, die sich mit den Synergien zwischen beiden Themenbereichen befassen.\\
In \citetitle{Review2018} (\cite{Review2018}) werden \ac{IOT}-Szenarien vorgestellt, die mit Blockchain umsetzbar sind. Es werden Anwendungsfälle aus den Bereichen Gesundheit, Logistik und Smart-City aufgezeigt. Darüber hinaus gehen die Autoren \citeauthor{Review2018} auf die praktischen Limitierungen ein, die durch Blockchain-Lösungen entstehen und beschreiben, an welchen Stellen weitere Forschung betrieben werden muss. Das Resultat ist, dass die Autoren keine allgemeingültige Blockchain-Lösung für \ac{IOT}-Anwendungsfälle identifizieren konnten, die Technologie aber ihrer Meinung nach großes Potential mit sich führe.\\
Eine ausführliche Untersuchung vieler Konsensalgorithmen hinsichtlich Tauglichkeit für \ac{IOT} wurde von \citeauthor{Salimitari2018ASO} in \citetitle{Salimitari2018ASO} (\cite{Salimitari2018ASO}) durchgeführt. Darüber hinaus werden einige der bekanntesten DLT-Implementierungen wie Hyperledger Fabric, Corda, Ethereum, Bitcoin und weitere gegenübergestellt und hinsichtlich Skalierbarkeit, Durchsatz, Latenz und weiteren untersucht. Es werden gewünschte Eigenschaften identifiziert, die eine \ac{DLT} mitbringen sollte, um für Anwendungsfälle im \ac{IOT}-Umfeld geeignet zu sein. Die Autoren \citeauthor{Salimitari2018ASO} unterscheiden nicht nach unterschiedlichen Anforderungen verschiedener Anwendungsfälle sondern betrachten \ac{IOT} als Gesamtes. Das Fazit lautet, dass es derzeit keine konkrete \ac{DLT}-Implementierung gebe, die alle Anforderungen von \ac{IOT} vollumfänglich erfülle. Es müsse ein bestehender Konsensalgorithmus erweitert oder die Vorteile verschiedener Implemetierungen kombiniert werden.\\
In \citetitle{BaaS2016} (\cite{BaaS2016}) wird untersucht, wo Blockchain-Knoten in einem \ac{IOT}-Umfeld gehostet werden können. Die Autoren \citeauthor{BaaS2016} beschreiben, dass die Umsetzung eines Blockchain-Knoten auf einem \ac{IOT}-Device aufgrund von fehlender Rechenleistung, hohem Stromverbrauch und geringer Bandbreite keine ratsame Lösung sei. Es wird die Umsetzung mittels Cloud- oder Fog-Computing vorgeschlagen. Als Ergebnis von Performance-Messungen kommen die Autoren zu dem Schluss, dass das Fog-Computing eine bessere Lösung darstelle als die Cloud-Variante.\\
Die Autoren \citeauthor{Eval2018} von \citetitle{Eval2018} (\cite{Eval2018}) sind der Überzeugung, dass Konsensalgorithmen, die auf dem Byzantinischen Fehler basieren, Potential bieten, um hoch-skalierende Anwendungen zu betreiben und geeignet für \ac{IOT} zu sein. Dazu untersuchen sie verschiedene \ac{BFT} Konsensalgorithmen und führen Performance-Tests durch. Die abschließende Erkenntnis der Autoren ist, dass weitere Forschung zur Lösung des Konsensproblems im \ac{IOT}-Umfeld nötig sei.\\
Nach den Autoren \citeauthor{convergence2019} von \citetitle{convergence2019} (\cite{convergence2019}) können \ac{IOT} und \ac{DLT} voneinander profitieren, man müsse jedoch zunächst Lösungen für Probleme wie den hohen Ressourcenverbrauch, zeitliche Verzögerungen, Bandbreite und Weitere lösen. Dazu beschreiben sie die einzelnen Komponenten sowie gängige Konsensverfahren einer Blockchain. Es werden Konsensverfahren und bestehende \acp{DLT} vorgestellt und nach Aspekten wie Durchsatz, Latenz, Sicherheit und Ressourcenverbrauch untersucht. Die Autoren unterscheiden nicht in unterschiedliche \ac{IOT}-Anwendungsfälle und kommmen zu dem Ergebnis, dass
einheitliche Protokolle für \ac{IOT}-Anwendungen auf \acp{DLT} definiert werden müssen und weitere Forschung besonders im Umfeld Konsensmechanismen notwendig sei.
%\cite{REYNA2018173}
