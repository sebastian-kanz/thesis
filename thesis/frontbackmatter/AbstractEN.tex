%*******************************************************
% Abstract in English
%*******************************************************
\pdfbookmark[1]{Abstract}{Abstract}


\begin{otherlanguage}{american}
	\chapter*{Abstract}
	The Internet of Things (IOT) gets more and more part of the everyday life. Smart-Home solutions, latency-sensitive connected cars or the vision of a smart-city affect the research works in their respective areas. The aim is a fully-automated machine-to-machine (M2M) handling of processes to automate and simplify our everday life. This leads to a huge amount of data needed to processed, transmitted and stored. With the introduction of the new mobile communication standard 5G and increasingly powerful devices the transmission and processing of data is guaranteed. The question where this huge amount of data is stored and further processed is still unanswered.\\
	Besides the topic of distributed ledger technologies (DLT) gets more attention as new use cases arise which can profit from the distributed infrastructure, the trustless environment and the decentralization. It makes sense to investigate to see if the yet relativly young technologies IOT and DLT can complement each other and where there may be synergies.\\
	This master thesis evaluates a selection of established DLTs for their suitability for the IOT environment with focus on the M2M part. The key research object is the thesis, that DLT is a suitable technology for IOT and the non-functional requirements for all DLT-IOT use cases are identical. For this purpose an IOT use case will be created which will be used for further analysis. Using a classification model requirements are identified that a DLT must satisy in order to be suitable for the use case. The criteria are applied, evaluated and ranked to a selection of DLT implementations. As a result of the analysis Ethereum is the most suitable solutio and therefore a prototypical implementation is made to check the result of the requirements evaluation and finally the research questions.\\
	The result is a structured and comprehensible rating of multiple established DLTs and their suitability for IOT use cases as well as a DLT based prototype inspired by a real use case which serves as a verification for the developed rating. It is shown that IOT can profit from DLT under certan circumstances. The findings of this work are, that DLT is a suitable technology for decentralized and asynchronous iot use cases where multiple distrusting parties are involved. The base of all non-functional requirements are identical for all DLT-IOT usecases.
\end{otherlanguage}
